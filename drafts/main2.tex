\documentclass{article}
\usepackage{times}
\usepackage{natbib}[round]
\usepackage[titletoc]{appendix}
\usepackage{graphicx}
\usepackage{lineno}
\usepackage{multirow}
\usepackage[english]{babel}
\usepackage{typearea} 
\usepackage{amssymb}
\usepackage{amsfonts}
\usepackage{amsmath}
\usepackage{enumerate}
\usepackage{mathtools}
\usepackage{graphicx}
\usepackage{wrapfig}
\usepackage{lscape}
\usepackage{rotating}
\newcommand{\Fig}[1]{Figure~\ref{fig:#1}}

\renewcommand{\baselinestretch}{1.5}
\newcommand{\bbar}[1]{\overline{#1}}

\newcommand\scalemath[2]{\scalebox{#1}{\mbox{\ensuremath{\displaystyle #2}}}}
\renewcommand{\familydefault}{\sfdefault}

\usepackage[font={small},labelfont={bf},justification=justified,margin=0.5cm]{caption}

\renewcommand{\thesection}{}
\renewcommand{\thesubsection}{\arabic{section}.\arabic{subsection}}

\usepackage{color}
	 \definecolor{darkred}{rgb}{0.75,0,0}
%	 \definecolor{darkgreen}{rgb}{0,0.5,0}
	 \definecolor{darkblue}{rgb}{0,0,0.75}
%	 \definecolor{magenta}{rgb}{0,0,0.75}
\newcommand{\cha}[1]{\textcolor{darkblue}{(#1)}}
\newcommand{\phg}[1]{\textcolor{darkred}{(#1)}}


\usepackage{hyperref}
\definecolor{darkgreen}{rgb}{0.1,0.6,0.3}
\definecolor{darkred}{rgb}{0.6,0.3,0.1}
\hypersetup{
    colorlinks=true,       % false: boxed links; true: colored links
    linkcolor=blue,          % color of internal links (change box color with linkbordercolor)
    citecolor=darkgreen,        % color of links to bibliography
    filecolor=magenta,      % color of file links
    urlcolor= black           % color of external links
}


\title{\vspace*{-22mm}\bf Multiple infections and complex life cycles}
%\author{Vaibhvi$^{1}$,
%Chaitanya S. Gokhale$^{2*}$\\
%\normalsize $^1$Molecular Physiology Group, \\
%\normalsize University of Kiel, \\
%\normalsize Am Botanisch Garten 3-9, D-24118 Kiel, Germany.\\
%\normalsize $^2$Research Group for Theoretical Models of Eco-evolutionary Dynamics, \\
%\normalsize Department of Evolutionary Theory, Max Planck Institute for Evolutionary Biology, \\
%\normalsize August-Thienemann-Stra{\ss}e 2, 24306 Pl\"{o}n, Germany.\\
%\normalsize $^{*}$gokhale@evolbio.mpg.de
%}

\date{}

\begin{document}

\linenumbers
\maketitle


\begin{abstract}
Abstract
\end{abstract}


\noindent
Keywords: a,b,c,d


\tableofcontents

\section{Introduction}

Life on earth is ubiquitously infested with parasite, many with complex life cycles \citep{zimmer:book:2001}.
While a complex life cycle can be defined as abrupt ontogenic changes in morphology and/or ecology \citep{Benesh:2016dj}, a complex parasitic life cycle typically involves numerous hosts that a parasite needs to traverse in the process of completing its life cycle.
Helminths are a prime example of how a successive transmission between multiple host species is necessary for developing the next generation.
The worms occupy different niches (hosts) in different stages of their lifecycle, moving through intermediate hosts until reaching a definitive host in which reproduction can finally occur.
Numerous factors determine the transmissibility and infectivity of the parasites along the tropic level of hosts \citep{froelick:PRSB:2021}.
Comparative growth rate, eventual body size, cost-benefits of host-manipulation and the probability of eventually finding a mate all play a major role in the development and maintenance of the life-history strategies of such parasites \citep{parker:Nature:2003,hammerschmidt:Evolution:2009} \cha{more references?}.
An integral part of all the above factors is the comparative approach.
Parasite often co-occur in the same host and this allows for complex within-host interactions since the evolutionary effect of the actions is not realised till reproduction occurs in the definitive host \citep{Hafer:2015gl}.

\cha{Discuss the work by \citep{alizon:AmNat:2008,gandon:Evolution:2009} 
What is done in those studies and what is the gap that exists. How do we fill in exactly that gap!}


In this manuscript we focus on dissecting the fundamental trade-off between transmissibility and host-manipulation when multiple hosts are present in a host and its evolutionary outcome in the associated trait space.

Theoretical work suggests that enhancer parasites should not exist because it disrupt the prey-predator coexistence (ref). 


- General introduction of parasite manipulation, definition, examples.

Life on earth is ubiquitously infectes with parasites, many with complex life cycles \citep{zimmer:book:2001}. While a complex life cycle can be defined as abrupt ontogenic changes in morphology and/or ecology \citep{Benesh:2016dj}, a complex parasitic life cycle typically involves numerous hosts that a parasite needs to traverse in the process of completing its life cycle. This results in the evolution of various strategies that enable the success of parasite transmission from one host to another. One famous strategy that inspires many scifi movies and novels is parasite manipulation, where a parasite is able to manipulate the morphology and/or behaviour of its intermediate host in order to enhance its transmission to its definitive host \citep{Hughes2012}. Parasite manipulation has been shown in many host-parasite systems \citep{Hughes2012, molyneux_jefferies1986}. Sand flies infected by \textit{Leishmania} parasites bite more and take more time for a blood meal from mammals (the definitive host of \textit{Leishmania}) compared to their uninfected counterparts \citep{ Rogers2007}. Copepods infected by cestode parasites are more active and easier to caught by sticklebacks (the definitive hosts of the cestodes) compared to uninfected copepods \citep{Wedekind1996}. 

- Work that has been done on parasite manipulation: empirical, theoretical.


- What is missing in the work of parasite manipulation: multiple infections, trophically transmitted parasite that has a free-living stage.

- What our model does and main results.


\section{Model}
We focus on the complex lifecycle of a trophically transmitted parasite. 
Thus the parasites can move through multiple hosts and reproduce inside their definitive hosts before being released into the environment. 
The parasites pass through the intermediate hosts between the environment and the definitive host.
When a definitive host consumes an infected intermediate host, the definitive host gets infected, and the parasite completes its lifecycle.

For simplicity, intermediate and definitive hosts can be infected by one (single infection) or at most two parasites (double infections). 
The probability that two parasites in the parasite pool co-transmit to an intermediate host is denoted by  $p$, and thus $1-p$ is the probability that a single parasite enters an intermediate host. 
When a definitive host consumes an intermediate host infected by two parasites, there is a probability $q$ that both parasites co-transmit to the definitive host.
With probability $1-q$, only one parasite successfully transmits. 
This formulation assumes that infection always happens whenever there are encounters between parasites and hosts.
The dynamics of a complex lifecycle parasite that requires two hosts is described by the following ODEs, firstly for the intermediate host as,
%
\begin{align}
\frac{dI_s}{dt} &= R(I_s, I_w, I_{ww}) - d I_s - \Pi_s(D_s, D_w, D_{ww}) I_s  - \eta_w  I_s \nonumber \\ 
\frac{dI_w}{dt} &=  (1 - p) \eta_w I_s  - (d + \alpha_w) I_w - \Pi_w(D_s, D_w, D_{ww}, \beta_w) I_w \label{odes:ihosts} \\
\frac{dI_{ww}}{dt} &= p \eta_w I_s  - (d + \alpha_ww) I_{ww} - \Pi_{ww}(D_s, D_w, D_{ww}, \beta_{ww}) I_{ww} \nonumber
\end{align}
%
where $R(I_s, I_w, I_{ww})$ represents the birth rate of the intermediate hosts, which is a function of both infected and uninfected individuals. 
%In other words, all types of intermediate hosts can reproduce. 
$\Pi_i$, where $i = \{s, w, ww\}$ is the predation function of definitive hosts on susceptible, singly infected and doubly infected intermediate hosts respectively. 
The predation function depends on the density of the definitive hosts and the manipulative strategies of parasites in the intermediate hosts. 
In particular, if a single parasite infects an intermediate host, the manipulation strategy is $\beta_w$. 
If two parasites infect it, the manipulation strategy is $\beta_{ww}$. 
The link between $\beta_w$ and $\beta_{ww}$, is explored further. 
The force of infection by parasites in the environment is denoted by $\eta_w = \gamma W$. 
The force of infection that corresponds respectively to singly infected intermediate host ($I_w$), or doubly infected intermediate hosts ($I_{ww}$) is denoted by $\lambda_i = \beta_i I_i$, where $i = \{ w, ww\}$. 

For the definitive hosts we have,
\begin{align}
\frac{dD_s}{dt} &= B(D_s,  D_w,  D_{ww},  I_s, I_w, I_{ww})  - \mu D_s - (\lambda_{ww} + \lambda_w) D_s \nonumber \\    
\frac{dD_w}{dt} &= (\lambda_w + 2 (1 - q) \lambda_{ww}) D_s - (\mu + \sigma_w) Dw - (2 (1 - q) \lambda_{ww} + \lambda_w) D_w  \label{odes:dhosts} \\         
\frac{dD_{ww}}{dt} &= q \lambda_{ww} D_s + (2 (1 - q) \lambda_{ww} + \lambda_w) D_w - (\mu + \sigma_{ww}) D_{ww} \nonumber
\end{align}
%
where $B(D_s, D_w, D_{ww}, I_s, I_w, I_{ww})$ represents the birth rate of definitive hosts, which is a function of population density of both intermediate and definitive hosts, infected or uninfected alike. 
The dynamics of the parasites in the environment are then given solely by,
\begin{align}
	\frac{dW}{dt} &= f_w D_w + f_{ww} D_{ww} - \delta W - \eta_w I_s \label{odes:eparasite}
\end{align}


Definitions of different parameters can be found in Table \ref{table:varpardescription}.
%
\begin{table}[!ht]
\begin{tabular}{|p{2.5cm}|p{12cm}|} 
\hline
Parameters and Variables    &  Description  \\
\hline
$I_i$  & Density of intermediate hosts that are susceptible $i=s$, singly infected $i=w$, or doubly infected $i=ww$ \\
\hline
$D_i$ & Density of definitive hosts that are susceptible $i=s$, singly infected $i=w$, or doubly infected $i=ww$ \\
\hline
$W$ & Density of parasites released from definitive hosts into the environment \\
\hline
$d$ & Natural death rate of intermediate hosts \\
\hline
$\alpha_i$ & Additional death rate of intermediate hosts due to infection by a single parasite ($i = w$) or two parasites ($i = ww$) \\
\hline
$p$ & Probability that two parasites cotransmit from the environment to an intermediate host \\
\hline
$\gamma$ & Transmission rate of parasites in the environment to intermediate hosts \\
\hline
$\mu$ & Natural death rate of definitive hosts \\
\hline
$\sigma_i$ & Additional death rate of definitive hosts due to infection by a single parasite ($i = w$) or two parasites ($i = ww$) \\
\hline
$\sigma_i$ & Additional death rate of the hosts due to being infected by a singly parasite ($i = w$) or two parasites ($i = ww$) \\
\hline
$q$ & Probability that two parasites cotransmit from intermediate hosts to definitive hosts \\
\hline
$\beta_i$ & Transmission rate of parasites from intermediate hosts to definitive hosts \\
\hline
$f_i$ & Reproduction rate of parasites in singly infected definitive hosts ($i = w$) or doubly infected hosts ($i = ww$)\\
\hline
$\delta$ & Natural death rate of parasites in the environment \\
\hline
\end{tabular}
\caption{Description of variables and parameters}
\label{table:varpardescription}
\end{table}

For simplicity, we assume that there is no sequential infection when parasites transmit from the environment to intermediate hosts. 
Sequential infection can happen when parasites transmit from intermediate hosts to definitive hosts. 
Therefore, a singly infected definitive host can be further infected by another parasite if it consumes infected intermediate hosts. 
The dynamics of the system are illustrated in figure (\ref{fig:schematic}).

\begin{figure}
\includegraphics[width=\textwidth]{Figures/schematic}
\caption{Schematic of the model}
\label{fig:schematic}
\end{figure}

\subsection{Ecological dynamics of the parasite}

System (\ref{odes:ihosts}, \ref{odes:dhosts}, \ref{odes:eparasite}) always has a parasite-free state, or disease-free state, as often used in epidemiological models, that is the parasite population is zero or $W = I_w = I_{ww} = D_w = D_{ww} = 0$. In this disease-free state, if a parasite is introduced, its population can spread when its reproduction ratio $R_0$ in this disease-free state is greater than one. The expression of the $R_0$ is as followed

\begin{align}
R_0 = & \gamma I_s \frac{ p q \beta_{ww}}{\alpha_{ww} + d + \Pi_{ww}} \frac{D_s}{\mu +\sigma_{ww}} \frac{f_{ww}}{\delta +\gamma I_s} + \nonumber \\
& \gamma  I_s \left( \frac{ (1-p)\beta_w}{\alpha_w + d + \Pi_w} + \frac{2 p (1-q) \beta_{ww}}{\alpha_{ww} + d + \Pi_{ww}} \right) \frac{D_s}{\lambda_w + 2 (1-q) \lambda_{ww}  + \mu + \sigma_w} \frac{f_w}{\delta +\gamma  I_s}
\end{align}

The expression of $R_0$ derived from the next generation method indicates the possible reproduction routes of a parasite, which can be via double infections or single infection (see supplementary ...).The first component corresponds to the double infections route, in which the focal parasite co-transmit with another parasite into a susceptible intermediate host, then co-transmit into a susceptible definitive host and reproduce. The second component corresponds to the single infection route, in which the focal parasite infects a susceptible intermediate host, either via singly or doubly infections. It then transmit alone into the susceptible definitive host, and eventually reproduce. It should be noted that, in a disease-free environment, parasites are so rare that the reproduction ratio compartments with sequential infection do not appear. 


When we introduce a parasite in the disease-free state, the parasite population can persist if its reproduction ratio $R_0|_{W = I_{w} = I_{ww} = D_{w} = D_{ww} = 0} > 1$. To solve this inequality, we need to specify the predation functions and the equilibrium value of $I_s$ and $D_s$. The explicit forms of the equilibrium of intermediate and definitive hosts depend largely on the birth functions $R$ and $B$ of respectively the intermediate and definitive host, as well as predation functions $\Pi_s, \Pi_w, \Pi_{ww}$. For simplicity, we consider linear functions for predation 
\begin{align*}
& \Pi_s(D_s + D_w + D_{ww}) = \rho (D_s + D_w + D_{ww}) \\
& \Pi_w(D_s, D_w, D_{ww}, \beta_w) = \beta_w (D_s + D_w + D_{ww}) \\
& \Pi_{ww}(D_s, D_w, D_{ww}, \beta_{ww}) =  \beta_{ww}) (D_s + D_w + D_{ww})
\end{align*}
Here $\rho$ is the capture rate when the uninfected intermediate hosts are caught by definitive hosts. If an intermediate hosts is infected, it is captured by the definitive hosts with rate $\beta_w$ if it is singly infected, and with rate $\beta_{ww}$ if it is doubly infected. Zero values for $\beta_w$ and $\beta_{ww}$ suggest that the parasite can manipulate the intermediate host to avoid the predation. The birth function of definitive hosts is also linear
\begin{align*}
B(D_s, D_w, D_{ww}, I_s, I_w, I_{ww}) = \rho c (D_s + D_w + D_{ww}) (I_s + I_w + I_{ww})
\end{align*}
where c is the efficiency of converting preys into offspring.

\subsubsection{Linear birth function of intermediate hosts}
We consider the system when the birth function $R$ is linear, that is, $R(I_s, I_w, I_ww) = r(I_s + I_w + I_ww)$. The equilibrium of intermediate and definitive hosts in the disease-free state are

\begin{align*}
& I_{s0}^* = \frac{\mu}{c \rho} \\
& D_{s0}^* = \frac{r - d}{\rho}
\end{align*}

This equilibrium is always unstable. 
We always observe cyclic behaviour of the equilibrium because, at this equilibrium, the jacobian matrix of the system (\ref{odes:ihosts}, \ref{odes:dhosts}, \ref{odes:eparasite}) always has one imaginary eigenvalue with a positive real part. 
This follows from the Lotka-Voltera system using linear functions for prey birth and predation (reference...). 
Because the disease-free dynamics is cyclic, it is difficult to analyse the spread of a parasite (often evaluated when the disease-free state is stable). 
Here, even if we solve the inequality $R_0 > 1$, which happens when the transmission rate from the environment to intermediate hosts $\gamma$ is greater than a threshold (the expression of the threshold is too complicated, hence it is not useful to write it here). 
In addition, the reproduction of the parasites has to be sufficiently large (again, the expression of the thresholds are too complicated such that it is useless to write it here).

Our simulations show that the parasite cannot persist even when its reproduction ratio is greater than one (Figure S\ref{fig:diseasefree:linear}). This result is, however, in agreement with the conclusion in \citet{Ripa:Evol:2013}, which suggests that it is harder for a mutant to invade a cyclic population. 
In our case, it is not the invasion of a mutant but a specific parasite in a cyclic disease-free host population. 
This issue deserves a more thorough investigation. 
To obtain a stable disease circulation state, we use a non-linear birth function of intermediate hosts. The following sections focus on analysing the ecological dynamics of the complex lifecycle parasite under different scenario of its manipulative ability.

\begin{figure}
\includegraphics[width=\textwidth]{Figures/diseasefree_linear}
\caption{Disease-free equilibrium using linear birth function. Parameter values  $\rho = 1.2, \  d = 0.9, \  r = 2.5, \ \gamma = 2.9, \ \alpha_w =  \alpha_{ww} =  0, \ \beta_w  = 1.5, \ \beta_{ww} = 1.5, \ p = 0.1,  \ c = 1.4, \ \mu = 0.9,  \ \sigma_w = \sigma_{ww} = 0, \ q = 0.01, \  f_w = 6.5, \  f_{ww} = 7.5, \ \delta = 0.9$ \textcolor{red}{put in supplementary}} 
\label{fig:diseasefree:linear}
\end{figure}

\subsubsection{Non-linear birth function of intermediate hosts}
The non-linear birth function of intermediate hosts is as followed

\begin{align*}
R(I_w, I_s,I_{ww}) = r (I_s + I_w + I_{ww}) (1 - k (I_s + I_w + I_{ww}))
\end{align*}

where $k$ is the intraspecific competition coefficient. The disease-free equilibrium is as follows

\begin{align*}
& I_s = \frac{\mu}{c \rho } \\
& D_s = \frac{c \rho  (r-d) - k \mu  r}{c \rho ^2}
\end{align*}

This equilibrium is stable if,

\begin{align*}
& r > d \\
& \frac{2 c \rho  \left(\sqrt{\frac{-d+\mu +r}{\mu }}-1\right)}{r}\leq k < \frac{c \rho  (r-d)}{\mu  r} \\
& \mu >\frac{4 c^2 \rho ^2 r - 4 c^2 d \rho ^2}{4 c k \rho r + k^2 r^2}
\end{align*}

The above conditions suggest that the intrinsic reproduction of intermediate hosts $r$ needs to be greater than their natural mortality rate $d$. 
More importantly, the intraspecific competition coefficient has to be within a range. 
It is neither too small such that the population cannot grow to infinity nor too large such that the population cannot survive. 
Finally, the natural mortality rate of the definitive host has to be sufficiently large. Satisfying such conditions, we obtain a stable disease-free equilibrium (Figure S\ref{fig:diseasefree:nonlinear}).

\begin{figure}[!ht]
\includegraphics[width=\textwidth]{Figures/diseasefree_nonlinear}
\caption{Disease-free equilibrium. \textcolor{red}{put in supplementary}}
\label{fig:diseasefree:nonlinear}
\end{figure}

When a parasite is introduced in the disease-free equilibrium, it can spread if its reproduction ratio $R_0^{res} > 1$. 
Since the expression is complicated, we could not obtain solutions for this inequality without assumptions. 
Assuming that double infections and single infection result in the same parasite virulence and parasite reproduction, that is, $\alpha_w = \alpha_{ww}$, $\sigma_w = \sigma_{ww}$,and $f_w = \epsilon f_{ww}$, we found the the parasite can establish and spread in the population of intermediate and definitive hosts if its reproduction value in single infection $f_w$ is greater than a threshold (the expression of the threshold is rather complicated, therefore it is not useful to write down its expression) (Figure \ref{fig:bifurfw:nonlinear}). Interestingly, if the reproduction rate of the parasite in double infection state is greater than that in single infection state, bistability can occur such that the parasite population will crash if it is disturbed an become too small (Figure \ref{fig:bistability}). 

\begin{figure}[!ht]
\includegraphics[width = \textwidth]{Figures/bistability.jpg}
\caption{A) Bifurcation graph of $\epsilon$ and $f_w$. The white area is where the parasite goes extinct. The vertical line indicates where $\epsilon = 1$, i.e. reproduction in singly infection is equal to reproduction in double infection. B) Bistability with $\epsilon = 2$. C) No bistability when $\epsilon = 1$. Blue circles indicate stable equilibrium, Yellow circles indicate unstable equilibrium, Yellowish-blue circles indicate bistability of the system. Parameter $\rho = 1.2, \  d = 0.9, \  r = 2.5, \ \gamma = 2.9, \ \alpha_w = 0, \ \alpha_ww =  0, \ \beta_w = 1.5, \ \beta_{ww} = 1.5, \ p = 0.1, \  c = 1.4, \ \mu = 3.9,  \ \sigma_w = 0, \ \sigma_{ww} = 0, \  q = 0.01, \ \delta = 0.9, \ k = 0.26$
}
\label{fig:bistability}
\end{figure}

Cooperation in parasite manipulation in fact widen the bistable state of the system (\ref{fig:manipbifur}). 
\begin{figure}[!ht]
\includegraphics[width=\textwidth]{Figures/manip_bifurcation.jpg}
\caption{Bifurcation graph of $\beta_w$ and $\beta_{ww}$. The white area is where the parasite goes extinct. The thick line indicates $\beta_w = \beta_ww$, i.e. manipulation in single infection is the same as manipulation in double infection. Blue circles indicate stable equilibrium, Yellowish-blue circles indicate bistability. A) When reproduction of double infection is smaller than that of single infection ($\epsilon = 0.5, f_w = 36$), B) When there is no difference in reproduction between single infection and double infection ($\epsilon = 1, f_w = 36$), C) When reproduction in double infection is greater than that of single infection ($\epsilon = 2, f_w = 35$).  Common parameter:$\rho = 1.2, \ d = 0.9, \ r = 2.5, \ \gamma = 2.9, \ \alpha_w = 0, \ \alpha_{ww} = 0, \ p = 0.1, \ c = 1.4, \ \mu = 3.9, \ \sigma_w = 0, \ \sigma_ww = 0, \ q = 0.01, \ \delta = 0.9, \ k = 0.26, \ \epsilon = 0.5$ }
\label{fig:manipbifur}
\end{figure}


\section{Discussion}

\textbf{Code availability}.
Appropriate {\tt{xyz}} computer code describing the model is available at {\url{https://github.com/tecoevo/xyz}}.



\bibliographystyle{plainnat}
\bibliography{references.bib}


\appendix


\end{document}