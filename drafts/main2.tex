\documentclass{article}
\usepackage{times}
\usepackage{natbib}[round]
\usepackage[titletoc]{appendix}
\usepackage{graphicx}
\usepackage{lineno}
\usepackage{multirow}
\usepackage[english]{babel}
\usepackage{typearea} 
\usepackage{amssymb}
\usepackage{amsfonts}
\usepackage{amsmath}
\usepackage{enumerate}
\usepackage{mathtools}
\usepackage{graphicx}
\usepackage{wrapfig}
\usepackage{lscape}
\usepackage{rotating}
\newcommand{\Fig}[1]{Figure~\ref{fig:#1}}

\renewcommand{\baselinestretch}{1.5}
\newcommand{\bbar}[1]{\overline{#1}}

\newcommand\scalemath[2]{\scalebox{#1}{\mbox{\ensuremath{\displaystyle #2}}}}
\renewcommand{\familydefault}{\sfdefault}

\usepackage[font={small},labelfont={bf},justification=justified,margin=0.5cm]{caption}

\renewcommand{\thesection}{}
\renewcommand{\thesubsection}{\arabic{section}.\arabic{subsection}}

\usepackage{color}
	 \definecolor{darkred}{rgb}{0.75,0,0}
%	 \definecolor{darkgreen}{rgb}{0,0.5,0}
	 \definecolor{darkblue}{rgb}{0,0,0.75}
%	 \definecolor{magenta}{rgb}{0,0,0.75}
\newcommand{\cha}[1]{\textcolor{darkblue}{(#1)}}
\newcommand{\phg}[1]{\textcolor{darkred}{(#1)}}


\usepackage{hyperref}
\definecolor{darkgreen}{rgb}{0.1,0.6,0.3}
\definecolor{darkred}{rgb}{0.6,0.3,0.1}
\hypersetup{
    colorlinks=true,       % false: boxed links; true: colored links
    linkcolor=blue,          % color of internal links (change box color with linkbordercolor)
    citecolor=darkgreen,        % color of links to bibliography
    filecolor=magenta,      % color of file links
    urlcolor= black           % color of external links
}


\title{\vspace*{-22mm}\bf Multiple infections and complex life cycles}
%\author{Vaibhvi$^{1}$,
%Chaitanya S. Gokhale$^{2*}$\\
%\normalsize $^1$Molecular Physiology Group, \\
%\normalsize University of Kiel, \\
%\normalsize Am Botanisch Garten 3-9, D-24118 Kiel, Germany.\\
%\normalsize $^2$Research Group for Theoretical Models of Eco-evolutionary Dynamics, \\
%\normalsize Department of Evolutionary Theory, Max Planck Institute for Evolutionary Biology, \\
%\normalsize August-Thienemann-Stra{\ss}e 2, 24306 Pl\"{o}n, Germany.\\
%\normalsize $^{*}$gokhale@evolbio.mpg.de
%}

\date{}

\begin{document}

\linenumbers
\maketitle


\begin{abstract}
Abstract
\end{abstract}


\noindent
Keywords: a,b,c,d


\tableofcontents

\section{Introduction}

Life on earth is ubiquitously infested with parasite, many with complex life cycles \citep{zimmer:book:2001}.
While a complex life cycle can be defined as abrupt ontogenic changes in morphology and/or ecology \citep{Benesh:2016dj}, a complex parasitic life cycle typically involves numerous hosts that a parasite needs to traverse in the process of completing its life cycle.
Helminths are a prime example of how a successive transmission between multiple host species is necessary for developing the next generation.
The worms occupy different niches (hosts) in different stages of their lifecycle, moving through intermediate hosts until reaching a definitive host in which reproduction can finally occur.
Numerous factors determine the transmissibility and infectivity of the parasites along the tropic level of hosts \citep{froelick:PRSB:2021}.
Comparative growth rate, eventual body size, cost-benefits of host-manipulation and the probability of eventually finding a mate all play a major role in the development and maintenance of the life-history strategies of such parasites \citep{parker:Nature:2003,hammerschmidt:Evolution:2009} \cha{more references?}.
An integral part of all the above factors is the comparative approach.
Parasite often co-occur in the same host and this allows for complex within-host interactions since the evolutionary effect of the actions is not realised till reproduction occurs in the definitive host \citep{Hafer:2015gl}.

\cha{Discuss the work by \citep{alizon:AmNat:2008,gandon:Evolution:2009} 
What is done in those studies and what is the gap that exists. How do we fill in exactly that gap!}


In this manuscript we focus on dissecting the fundamental trade-off between transmissibility and host-manipulation when multiple hosts are present in a host and its evolutionary outcome in the associated trait space.

\section{Model}
We focus on the complex lifecycle of a trophically transmitted parasite. 
Parasites reproduce inside their definitive hosts and are released into the environment. 
They infect intermediate hosts when the hosts encounter the parasite pool. 
When an infected intermediate host is consumed by a definitive host, the definitive host gets infected and the parasite complete its lifecycle.

For simplicity, intermediate and definitive hosts can be infected by one (single infection) or at most two parasites (double infections). Assuming that infection always happens whenever there are encounters between parasites and uninfected hosts, $p$ is the probability that two parasites in the parasite pool co-transmit to an intermediate host, and $1-p$ is the probability that a single parasites enters an intermediate host. 
When a definitive host consumes an intermediate host infected by two parasites, there is a probability $q$ that both parasites co-transmit to the definitive host, and $1-q$ is the probability that only one parasite successfully transmits. 
The dynamics of a complex lifecycle parasite that requires two hosts is described by the following ODEs, firstly for the intermediate host as,

\begin{align}
\frac{dI_s}{dt} &= R(I_s, I_w, I_{ww}) - d I_s - \Pi_s(D_s, D_w, D_{ww}) I_s  - \eta_w  I_s \nonumber \\ 
\frac{dI_w}{dt} &=  (1 - p) \eta_w I_s  - (d + \alpha_w) I_w - \Pi_w(D_s, D_w, D_{ww}, \beta_w) I_w \label{odes:ihosts} \\
\frac{dI_{ww}}{dt} &= p \eta_w I_s  - (d + \alpha_ww) I_{ww} - \Pi_{ww}(D_s, D_w, D_{ww}, \beta_{ww}) I_{ww} \nonumber
\end{align}

where  $R(I_s, I_w, I_{ww})$ represents the birth rate of the intermediate hosts, which is a function of both infected and noninfected individuals. 
In other words, all types of intermediate hosts can reproduce. 
$\Pi_i$, where $i = \{s, w, ww\}$ is the predation function of definitive hosts on respectively, susceptible, singly infected and doubly infected intermediate hosts. 
The predation function depends on the density of the definitive hosts and the manipulative strategies of parasites in the intermediate hosts. 
In particular, if an intermediate host is infected by a single parasite, the manipulation strategy is $\beta_w$, and if it is infected by two parasites, the manipulation strategy is $\beta_{ww}$. 
There is a link between $\beta_w$ and $\beta_{ww}$, but we need not specify the link for now. 
$\eta_w = \gamma W$ is the force of infection by parasites in the environment. 
$\lambda_i = \beta_i I_i$, where $i = \{ w, ww\}$, is the force of infection that corresponds respectively to singly infected intermediate host ($I_w$), or doubly infected intermediate hosts ($I_{ww}$). 

For the definitive hosts we have,
\begin{align}
\frac{dD_s}{dt} &= B(D_s,  D_w,  D_{ww},  I_s, I_w, I_{ww})  - \mu D_s - (\lambda_{ww} + \lambda_w) D_s \nonumber \\    
\frac{dD_w}{dt} &= (\lambda_w + 2 (1 - q) \lambda_{ww}) D_s - (\mu + \sigma_w) Dw - (2 (1 - q) \lambda_{ww} + \lambda_w) D_w  \label{odes:dhosts} \\         
\frac{dD_{ww}}{dt} &= q \lambda_{ww} D_s + (2 (1 - q) \lambda_{ww} + \lambda_w) D_w - (\mu + \sigma_{ww}) D_{ww} \nonumber
\end{align}
%
where $B(D_s, D_w, D_{ww}, I_s, I_w, I_{ww})$ represents the birth rate of definitive hosts, which is a function of population density of both intermediate and definitive hosts, infected or non-infected alike. 
The dynamics of the parasites in the environment are then given solely by,
\begin{align}
	\frac{dW}{dt} &= f_w D_w + f_{ww} D_{ww} - \delta W - \eta_w I_s \label{odes:eparasite}
\end{align}


Definitions of different parameters can be found in Table \ref{table:varpardescription}.

\begin{table}[!ht]
\begin{tabular}{|p{2.5cm}|p{12cm}|} 
\hline
Parameters and Variables    &  Description  \\
\hline
$I_i$  & Density of intermediate hosts that are susceptible $i=s$, singly infected $i=w$, or doubly infected $i=ww$ \\
\hline
$D_i$ & Density of definitive hosts that are susceptible $i=s$, singly infected $i=w$, or doubly infected $i=ww$ \\
\hline
$W$ & Density of parasites released from definitive hosts into the environment \\
\hline
$d$ & Natural death rate of intermediate hosts \\
\hline
$\alpha_i$ & Additional death rate of intermediate hosts due to infection by a single parasite ($i = w$) or two parasites ($i = ww$) \\
\hline
$p$ & Probability that two parasites cotransmit from the environment to an intermediate host \\
\hline
$\gamma$ & Transmission rate of parasites in the environment to intermediate hosts \\
\hline
$\mu$ & Natural death rate of definitive hosts \\
\hline
$\sigma_i$ & Additional death rate of definitive hosts due to infection by a single parasite ($i = w$) or two parasites ($i = ww$) \\
\hline
$\sigma_i$ & Additional death rate of the hosts due to being infected by a singly parasite ($i = w$) or two parasites ($i = ww$) \\
\hline
$q$ & Probability that two parasites cotransmit from intermediate hosts to definitive hosts \\
\hline
$\beta_i$ & Transmission rate of parasites from intermediate hosts to definitive hosts \\
\hline
$f_i$ & Reproduction rate of parasites in singly infected definitive hosts ($i = w$) or doubly infected hosts ($i = ww$)\\
\hline
$\delta$ & Natural death rate of parasites in the environment \\
\hline
\end{tabular}
\caption{Description of variables and parameters}
\label{table:varpardescription}
\end{table}

For simplicity, we assume that there is no sequential infection when parasites transmit from the environment to intermediate hosts. 
Sequential infection can happen when parasites transmit from intermediate hosts to definitive hosts. 
Therefore, a singly infected definitive host can be further infected by another parasite if it consumes infected intermediate hosts. The dynamics of the system are illustrated in figure (\ref{fig:schematic}).

\begin{figure}
\includegraphics[width=\textwidth]{Figures/schematic}
\caption{Schematic of the model}
\label{fig:schematic}
\end{figure}

\subsection{Ecological dynamics of the parasite}

System (\ref{odes:ihosts}, \ref{odes:dhosts}, \ref{odes:eparasite}) always has a parasite-free state, or disease-free state as often used in epidemiological models, that is the parasite population is zero or $W = I_w = I_{ww} = D_w = D_{ww} = 0$. In this disease-free state, if a parasite is introduced, its population can spread when its reproduction ratio $R^{res}_0$ in this disease-free state is greater than one. The expression of the $R^{res}_0$ is as followed

\begin{align}
R^{res}_0 = & \gamma I_s \frac{ p q \beta_{ww}}{\alpha_{ww} + d + \Pi_{ww}} \frac{D_s}{\mu +\sigma_{ww}} \frac{f_{ww}}{\delta +\gamma I_s} + \nonumber \\
& \gamma  I_s \left( \frac{ (1-p)\beta_w}{\alpha_w + d + \Pi_w} + \frac{2 p (1-q) \beta_{ww}}{\alpha_{ww} + d + \Pi_{ww}} \right) \frac{D_s}{\lambda_w + 2 (1-q) \lambda_{ww}  + \mu + \sigma_w} \frac{f_w}{\delta +\gamma  I_s}
\end{align}

The expression of $R^{res}_0$ derived from the next generation method indicates the possible reproduction routes of a parasite, which can be via double infections or single infection.The first component corresponds to the double infections route, in which the focal parasite co-transmit with another parasite into a susceptible intermediate host, then co-transmit into a susceptible definitive host and reproduce. The second component corresponds to the single infection route, in which the focal parasite infects a susceptible intermediate host, either via singly or doubly infections. It then transmit alone into the susceptible definitive host, and eventually reproduce. It should be noted that, in a disease-free environment, parasites are so rare that the reproduction ratio compartments with sequential infection do not appear. 


When we introduce a parasite in the disease-free state, the parasite population can persist if its reproduction ratio $R_0|_{W = I_w = I_ww = D_w = D_ww = 0} > 1$. To solve this inequality, we need to specify the predation functions and the equilibrium value of $I_s$ and $D_s$. The explicit forms of the equilibrium of intermediate and definitive hosts depend largely on the birth functions $R$ and $B$ of respectively the intermediate and definitive host, as well as predation functions $\Pi_s, \Pi_w, \Pi_ww$. For simplicity, we consider linear functions for predation 
\begin{align*}
& \Pi_s(D_s + D_w + D_{ww}) = \rho (D_s + D_w + D_{ww}) \\
& \Pi_w(D_s, D_w, D_{ww}, \beta_w) = (\rho + \beta_w) (Ds + Dw + Dww) \\
& \Pi_{ww}(D_s, D_w, D_{ww}, \beta_{ww}) = (\rho + \beta_{ww}) (Ds + Dw + Dww)
\end{align*}
Here $\rho$ is the baseline capture rate when the uninfected intermediate hosts are caught by definitive hosts. Manipulation strategy of the parasite is incorporated as an addition to the baseline capture rate $\rho$. The birth function of definitive hosts is also linear
\begin{align*}
B(D_s, D_w, D_{ww}, I_s, I_w, I_{ww}) = \rho c (D_s + D_w + D_{ww}) (I_s + I_w + I_{ww})
\end{align*}
where c is the efficiency of converting preys into offspring.

\subsubsection{Linear birth function of intermediate hosts}
We consider the system when the birth function $R$ is linear, that is, $R(I_s, I_w, I_ww) = r(I_s + I_w + I_ww)$. The equilibrium of intermediate and definitive hosts in the disease-free state are

\begin{align*}
& I_{s0}^* = \frac{\mu}{c \rho} \\
& D_{s0}^* = \frac{r - d}{\rho}
\end{align*}

This equilibrium is always unstable. 
We always observe cyclic behaviour of the equilibrium because, at this equilibrium, the jacobian matrix of the system (\ref{odes:ihosts}, \ref{odes:dhosts}, \ref{odes:eparasite}) always has one imaginary eigenvalue with a positive real part. 
This follows from the Lotka-Voltera system using linear functions for prey birth and predation. 
Because the disease-free dynamics is cyclic, it is difficult to analyse the spread of a parasite (often evaluated when the disease-free state is stable). 
Here, even if we solve the inequality $R_0 > 1$, which happens when the transmission rate from the environment to intermediate hosts $\gamma$ is greater than a threshold (the expression of the threshold is too complicated, hence it is not useful to write it here). 
In addition, the reproduction of the parasites has to be sufficiently large (again, the expression of the thresholds are too complicated such that it is useless to write it here).

Our simulations show that the parasite cannot persist even when its reproduction ratio is greater than one (Figure \ref{fig:diseasefree:linear}). This result is, however, in agreement with the conclusion in \citet{Ripa:Evol:2013}, which suggests that it is harder for a mutant to invade a cyclic population. 
In our case, it is not the invasion of a mutant but a specific parasite in a cyclic disease-free host population. 
This issue deserves a more thorough investigation. 
However, as we focus on the evolutionary aspect of parasite manipulation, we use a non-linear birth function of intermediate hosts to obtain stable equilibrium for the invasion analysis.

\begin{figure}
\includegraphics[width=\textwidth]{Figures/diseasefree_linear}
\caption{Disease-free equilibrium using linear birth function}
\label{fig:diseasefree:linear}
\end{figure}

\subsubsection{Non-linear birth function of intermediate hosts}
The non-linear birth function of intermediate hosts is as followed

\begin{align*}
R(I_w, I_s,I_{ww}) = r (I_s + I_w + I_{ww}) (1 - k (I_s + I_w + I_{ww}))
\end{align*}

where $k$ is the intraspecific competition coefficient. The disease-free equilibrium is as follows

\begin{align*}
& I_s = \frac{\mu}{c \rho } \\
& D_s = \frac{c \rho  (r-d) - k \mu  r}{c \rho ^2}
\end{align*}

This equilibrium is stable if,

\begin{align*}
& r > d \\
& \frac{2 c \rho  \left(\sqrt{\frac{-d+\mu +r}{\mu }}-1\right)}{r}\leq k < \frac{c \rho  (r-d)}{\mu  r} \\
& \mu >\frac{4 c^2 \rho ^2 r - 4 c^2 d \rho ^2}{4 c k \rho r + k^2 r^2}
\end{align*}

The above conditions suggest that the intrinsic reproduction of intermediate hosts $r$ needs to be greater than their natural mortality rate $d$. 
More importantly, the intraspecific competition coefficient has to be within a range. 
It is neither too small such that the population cannot grow to infinity nor too large such that the population cannot survive. 
Finally, the natural mortality rate of the definitive host has to be sufficiently large. Satisfying such conditions, we obtain a stable disease-free equilibrium (Figure \ref{fig:diseasefree:nonlinear}).

\begin{figure}
\includegraphics[width=\textwidth]{Figures/diseasefree_nonlinear}
\caption{Disease-free equilibrium}
\label{fig:diseasefree:nonlinear}
\end{figure}

When a parasite is introduced in the disease-free equilibrium, it can spread if its reproduction ratio $R_0^{res} > 1$. 
Since the expression is complicated, we could not obtain solutions for this inequality without assumptions. 
Assuming that double infections and single infection result in the same parasite virulence and parasite reproduction, that is, $\alpha_w = \alpha_{ww}$, $\sigma_w = \sigma_{ww}$,and $f_w = \epsilon f_{ww}$, we found the the parasite can establish and spread in the population of intermediate and definitive hosts if its reproduction value in single infection $f_w$ is greater than a threshold (the expression of the threshold is rather complicated, therefore it is not useful to write down its expression) (Figure \ref{fig:bifurfw:nonlinear}).

\begin{figure}
\includegraphics[width=\textwidth]{Figures/bifurcation_fw_NL.jpg}
\caption{Bifurcation graph of reproduction value in singly infection. $f_ww = \epsilon f_w$, where $\epsilon \leq 1$ means reproduction in double infection is smaller than or equal to reproduction in single infection. $\epsilon > 1$ means reproduction in double infection is greater than reproduction in single infection. Dots indicate stable equilibrium whereas stars indicate unstable equilibrium.}
\label{fig:bifurfw:nonlinear}
\end{figure}

Suppose reproduction in single-infection is small, but in double-infection is sufficiently large. 
In that case, the parasite population settles in the intermediate and definitive hosts (Figure \ref{fig:bifurepsilonfw:nonlinear}).

\begin{figure}
\includegraphics[width = \textwidth]{Figures/bifurcation_epsilonfw_NL.jpg}
\caption{Bifurcation graph of $\epsilon$ and $f_w$. The white area is where the parasite goes extinct. The vertical line indicates where $\epsilon = 1$, i.e. reproduction in singly infection is equal to reproduction in double infection. Dots indicate stable equilibrium whereas stars indicate unstable equilibrium.}
\label{fig:bifurepsilonfw:nonlinear}
\end{figure}

The transmission rate from intermediate hosts to definitive hosts, which is determined by parasite manipulation strategy, also play an essential role in the existence of the parasite population. 
If this transmission rate is too low, then the parasite goes extinct regardless of its reproduction value (Figure \ref{fig:bifurbetawfw:nonlinear}).

\begin{figure}
\includegraphics[width=\textwidth]{Figures/bifurcation_betawfw_NL.jpg}
\caption{Bifurcation graph of $\beta_w$ and $f_w$. The white area is where the parasite goes extinct. The vertical line indicates where $\beta_w = \beta_ww$, i.e. manipulation in single infection is the same as manipulation in double infection. Dots indicate stable equilibrium whereas stars indicate unstable equilibrium.}
\label{fig:bifurbetawfw:nonlinear}
\end{figure}

We then consider the evolution of the host manipulation strategy $\beta_w$. 
Assuming that mutation is so rare, a mutant with a manipulation strategy $\beta_m$ that is slightly different from the resident arises when the resident parasite is at its stable ecological equilibrium.

\subsection{Mutant dynamics}
Considering that a resident parasite with dynamics (\ref{odes:ihosts}, \ref{odes:dhosts}, \ref{odes:eparasite}) is at its non-zero equilibrium ($W^*, I_s^*, I_w^*, I_{ww}^*, D_s^*, D_w^*, D_{ww}^*$), the dynamics of a rare mutant parasite that enter the resident population will be

\begin{subequations}
\begin{align}
& \frac{dI_m}{dt} = (1 - p) \frac{M}{M + W^*} \eta I_s - (d + \alpha_m) I_m -\Pi_m(D_s^*, D_w^*,  D_{ww}^*,  \beta_m, D_m, D_{mm}, D_{mw}) I_m \\
& \frac{dI_{mm}}{dt} = p \frac{M^2}{(M + W^*)^2} \eta I_s - (d + \alpha_{mm}) I_{mm} - \Pi_{mm}(D_s^*, D_w^*, D_{ww}^*, \beta_{mm}, D_m, D_{mm}, D_{mw}) I_{mm}\\
& \frac{dI_{mw}}{dt} =  p \frac{2 M W^*}{(M + W^*)^2} \eta I_s - (d + \alpha_{mw}) I_{mw} -\Pi_{mw}(D_s^*, D_w^*, D_{ww}^*, \beta_{mw}, D_m, D_{mm}, D_{mw}) I_{mw} \\
& \frac{dD_m}{dt} = ( \lambda_m + (1 - q) (2 \lambda_{mm} + \lambda_{mw})) D_s - (\mu + \sigma_m) D_m - (\lambda_w + \lambda_m + (1 - q) (2 \lambda_{ww} + \lambda_{mw} + \lambda_{mm}) ) D_m \\
& \frac{dD_{mm}}{dt} = q \lambda_{mm} D_s + (\lambda_m + (1 - q)(2 \lambda_{mm} + \lambda_{mw})) D_m - (\mu + \sigma_{mm}) D_{mm} \\
& \frac{dD_{mw}}{dt} = q \lambda_{mw} D_s + (\lambda_m + (1 - q)(2 \lambda_{mm} + \lambda_{mw})) D_w + (\lambda_w + (1 - q) (2 \lambda_{ww} + \lambda_{mw})) D_m - (\mu + \sigma_{mw}) D_{mw} \\
& \frac{dM}{dt} = f_m D_m + f_{mm} D_{mm} +  f_{mw} D_{mw} - \delta M - \eta I_s;
\end{align}
\label{odes:mutdynamics}
\end{subequations}

where $\eta =\gamma (M + W^*)$ is the force of infection of parasites in the environment, which now includes both mutants and residents. $\lambda_m = \beta_m I_m$, $\lambda_{mm} = \beta_{mm} I_{mm}$, $\lambda_{mw} = \beta_{mw} I_{mw}$ are the force of infection of, respectively, intermediate hosts infected by a single mutant, two mutants, and a mix of mutant and resident. 

\subsection{Invasion fitness}
The invasion fitness of the mutant will be calculated using the next generation method \citep{hurford:JRSI:2010}.

The matrix $\mathbf{A}$ that describes the dynamics (\ref{odes:mutdynamics}) of the mutant can be written as $\mathbf{A} = \mathbf{F} - \mathbf{V}$, where $\mathbf{F}$ is the matrix that describes the contribution of a compartment to the next generation. 
Because we consider a complex lifecycle parasite, a new generation begins with the reproduction and release of parasites into the environment, and matrix $\mathbf{F}$ can be written as,

\begin{align*}
    \mathbf{F} = 
    \begin{pmatrix}
   	 0 & 0 & 0 & 0   & 0 	  & 0 	   & 0 \\
	 0 & 0 & 0 & 0   & 0 	  & 0 	   & 0 \\
	 0 & 0 & 0 & 0   & 0 	  & 0 	   & 0 \\
	 0 & 0 & 0 & 0   & 0      & 0	   & 0 \\
	 0 & 0 & 0 & 0   & 0      & 0      & 0 \\ 
	 0 & 0 & 0 & 0   & 0      & 0      & 0 \\
	 0 & 0 & 0 & f_m & f_{mm} & f_{mw} & 0
    \end{pmatrix}
\end{align*}

The matrix $\mathbf{V}$ describes the transitions from one compartment to the others and death rates of all compartments
\[
    \mathbf{V} = 
    \begin{pmatrix*}
     v_{11}  & 0 & 0 & 0 & 0 & 0 & v_{17} \\
     0  &  v_{22} & 0 & 0 & 0 & 0 & v_{27} \\ 
     0  &  0  &  v_{33} &  0  &  0  &  0  & v_{37} \\
    v_{41}  &  v_{42} & v_{43} &  v_{44} &  0 & 0 & 0 \\
     0 & v_{52}  &  0  & v_{54} &  v_{55} & 0 & 0  \\
     v_{61} & v_{62} & v_{63} & v_{64}  &  0  & v_{66} & 0 \\
     0  &  0  &  0  &  0  &  0  &  0  &  v_{77}
    \end{pmatrix*}
\]
where the entries of the $\mathbf{V}$ matrix can be found in Table \ref{table:Ventries}

\begin{table}[!ht]
\begin{tabular}{|c|c|}
\hline
Matrix entry & Expression \\
\hline
$v_{11}$ & $(d + \alpha_m) + \Pi_m(D_s^*, D_w^*, D_{ww}^*, \beta_m, D_m, D_{mm}, D_{mw}) $ \\
\hline
$v_{17}$ & $-(1 - p) \gamma I_s$ \\
\hline
$v_{22}$ &  $(d + \alpha_{mm}) + \Pi_{mm}(D_s^*, D_w^*, D_{ww}^*, \beta_{mm}, D_m, D_{mm}, D_{mw})$ \\
\hline
$v_{27}$ & $-p \gamma \frac{M}{M + W^*}  I_s$ \\
\hline
$v_{33}$ & $ (d + \alpha_{mw}) + \Pi_{mw}(D_s^*, D_w^*, D_{ww}^*, \beta_{mm}, D_m, D_{mm}, D_{mw})$ \\
\hline
$v_{37}$ & $- p \gamma 2 \frac{W}{M + W^*} I_s$ \\
\hline
$v_{41}$ & $-\beta_m D_s$ \\
\hline
$v_{42}$ & $-(1 - q) \beta_{mm} D_s$ \\
\hline
$v_{43}$ & $-(1 - q) \beta_{mw} D_s$ \\
\hline
$v_{44}$ & $(\mu + \sigma_m) + (\lambda_w + \lambda_m + (1 - q) (\lambda_{ww} + \lambda_{mw} + \lambda_{mm}) $ \\
\hline
$v_{52}$ & $- q \beta_{mm} D_s$ \\
\hline
$v_{54}$ & $-(\lambda_m + (1 - q) (\lambda_{mm} + \lambda_{mw}))$ \\
\hline
$v_{55}$ & $(\mu + \sigma_{mm})$ \\
\hline
$v_{61}$ & $-\beta_m D_w$ \\
\hline
$v_{62}$ & $ -(1 - q) \beta_{mm} D_w$ \\
\hline
$v_{63}$ & $- q \beta_{mw} D_s - (1 - q) \beta_{mw} D_w$ \\
\hline
$v_{64}$ &  $-(\lambda_w + (1 - q) (\lambda_{ww} +\lambda_{mw}))$ \\
\hline
$v_{66}$ & $(\mu + \sigma_{mw})$ \\
\hline
$v_{77}$ & $\delta + (1 + 2 \frac{W}{M + W^*} + \frac{M}{M + W^*}) \gamma I_s$ \\
\hline
\end{tabular}
\caption{Expressions of $\mathbf{V}$ matrix entries}
\label{table:Ventries}
\end{table}

According to the next generation method, the reproduction value $R_0$ of a mutant is the spectral radius of the next generation matrix $\mathbf{F.V^{-1}}$, evaluated when the mutant is extremely rare, i.e. $M = D_{m} = D_{mm} = D_{mw} = 0$. Here it is the unique nonzero eigenvalue of $\mathbf{F.V^{-1}}$. Similar to the expression of $R^{res}_0$ of a resident, the expression of $R_0$ of a mutant also shows two routes of reproduction: via single infection or via double infections. As double infections in the definitive host can be a result of co-transmission and sequential transmission with a resident, the reproduction ratio of a mutant is more complex than that of a resident. In particular, it contains four compartments $R_0 = R_{d1} + R_{d2} + R_{d3} + R_s$, where $R_{d1}$, $R_{d2}$, and $R_{d3}$ are compartments of double infections route, and $R_s$ is the compartment of single infection route. In particular,

\begin{align*}
R_{d1} = \gamma  I_s \left( \frac{(1-p) \beta_m}{\alpha_m + d + \Pi_m} + \frac{2 p (1-q) \beta_{mw}}{\alpha_{mw} + d + \Pi_{mw}} \right)\frac{D_w}{\mu + \sigma_{mw}} \frac{f_{mw}}{\delta +3 \gamma  I_s}
\end{align*}
 
corresponds to the compartment where the mutant from the pool infects a susceptible intermediate host via either single infection or co-transmission with a resident. It then infects a definitive host that is already occupied by a resident, i.e. the mutant arrives after the resident, and it eventually reproduces.

\begin{align*}
R_{d2} = \gamma  I_s \left(\frac{\beta_m (1-p)}{\alpha_m + d + \Pi_m} + \frac{2 p (1-q) \beta_{mw}}{\alpha_{mw} + d + \Pi_{mw}}\right) \frac{D_s (\lambda_w + 2  (1-q) \lambda_{ww})}{(\mu + \sigma_{mw}) (\lambda_w  + 2 (1-q) \lambda_{ww} + \mu + \sigma_m)} \frac{f_{mw}}{\delta +3 \gamma  I_s}
\end{align*}

corresponds to the compartment where similar to $R_{d1}$ the mutant from the pool infects a susceptible intermediate host via either single infection or co-transmission with a resident. It then infects a susceptible definitive host which is then infected by a resident, i.e. the mutant arrives before the resident. Finally, the mutant releases offspring via the double infected definitive host.

\begin{align*}
R_{d3} = \gamma  I_s \frac{2 p q \beta_{mw}}{\alpha_{mw} + d + \Pi_{mw}} \frac{D_s}{\mu + \sigma_{mw}} \frac{f_{mw}}{\delta +3 \gamma  I_s}
\end{align*}

corresponds to the co-transmission compartment where the mutant co-transmit with a resident from the parasite pool into a susceptible intermediate host, and then into a susceptible definitive host, where it finally reproduces.

\begin{align*}
R_s = \gamma  I_s \left(\frac{(1-p) \beta_m}{\alpha_m + d + \Pi_m} + \frac{2 p (1-q) \beta_{mw}}{\alpha_{mw} + d + \Pi_{mw}}\right) \frac{D_s}{\lambda_w  + 2 (1-q) \lambda_{ww} + \mu + \sigma_m}\frac{f_m}{\delta +3 \gamma  I_s}
\end{align*}

corresponds to the compartment where the mutant the mutant from the pool again infects a susceptible intermediate host via either single infection or co-transmission with a resident. It then singly infects a susceptible host and releases offspring into the parasite pool. It should be noted that there is no double infections with two mutants in the expression of $R_0$ because we assume that mutation is so rare that the populations of hosts doubly infected with two mutants vanish and do not have any significant effect on $R_0$. A mutant can invade if its basic reproduction ratio $R_0 > 1$. 

\subsection{Possible trade-offs}
In order to proceed the invasion analysis, we will introduce a trade-off between transmission and reproduction. Assuming that an underlying manipulative trait $x$ affects both transmission rate and reproduction rate such that the higher the manipulative ability, the higher the transmission rate is. However, manipulation consumes energy, thus, the higher the manipulation level, the lower the reproduction ability (reference?). The transmission rate from intermediate hosts to definitive hosts and the reproduction rates are now functions of the manipulative trait $x$. Furthermore,$\beta_w'(x) > 0$, $\beta_{ww}'(x) > 0$, whereas $f_w'(x) < 0$, and $f_{ww}'(x) < 0$.

In this article, we take into account multiple infections, therefore, manipulative ability when the parasite is alone will be completely different from when it is co-infected with another parasite. There are two scenarios of manipulation in multiple infections that we could consider: (i) Manipulation in double infections can be stronger than manipulation in single infection because the two parasites cooperate to increase their transmission rate, or (ii) Manipulation in double infections is weaker than manipulation in single infection because one parasite sabotage the transmission of the other parasite. 
\section{Discussion}

\textbf{Code availability}.
Appropriate {\tt{xyz}} computer code describing the model is available at {\url{https://github.com/tecoevo/xyz}}.



\bibliographystyle{plainnat}
\bibliography{references.bib}


\appendix


\end{document}