 \documentclass[10,DIN, pagenumber=false, parskip=half,fromalign=right, fromphone=false,fromemail=true, fromurl=false,fromlogo=true, fromrule=false]{scrlttr2}
\usepackage[mac]{inputenc}
\usepackage{graphicx}
\usepackage{float}
\usepackage{caption}
%\usepackage[ngerman]{babel}
\RequirePackage{graphicx}
\usepackage{hyperref}
\hypersetup{
    colorlinks=true,
    linkcolor=blue,
    filecolor=magenta,      
    urlcolor=blue,
}
\setkomafont{footnote}{\normalcolor \sfseries} 
 \setkomavar{date}{\vspace{-0.05cm}    \sffamily \today}
\setkomavar{fromname}{  \sffamily }
\setkomavar{fromaddress}{ \sffamily  
Dr. Phuong Linh Nguyen \\
Chemin du muse\'e 15\\
University of Fribourg\\
Fribourg, 1700 Switzerland
}
\setkomavar{fromlogo}{\includegraphics[scale=0.53]{Fribourg_Logo}}
\setkomavar{backaddress}{\sffamily Phuong L. Nguyen, Department of Biology, University of Fribourg, Switzerland} 
\setkomavar{fromemail}{\sffamily linh.phuong.nguyen@evobio.eu}

\setkomavar{subject}{Comments to the editor\\}
\newfloat{figure}{htbp}{figs}

\begin{document}
\sffamily

\begin{letter}{
\sffamily
\vspace{-0.4cm}
Dries Bonte \\
Editor-in-Chief\\
Oikos Journal
}
\opening{\sffamily \vspace{-1cm} Dear Prof. Bonte,}
\vspace{-0.3cm}

We submitted our manuscript, entitled: ``On multiple infections by parasites with complex life cycles." for consideration for publication as a Research paper in Oikos Journal in October 2023.
Our manuscript proposes a theoretical advance in the understanding of a parasitological phenomenon.
We received the comments from one reviewer at the end of February 2024.

Our model shows the importance of considering multiple infections a norm rather than an exception. 
For a parasite with a complex life cycle, we show that including its free-living state in the environment, often ignored in theory, allows for the straightforward modelling of multiple infections.
Furthermore, we show the stability of the multi-trophic parasite along with the predator-prey chain they go through during their development.
This involves a delicate balance between the manipulation of the hosts and the tradeoff with reproduction.
This novel result helps us understand the dynamics of parasites with complex lifecycles and their multi-host interactions in-depth. 
The reviewer has observed the value of our manuscript to the field and has helped us immensely with the comments on improving the readability of our manuscript.

We have addressed the reviewer's comments; however, we could have also benefitted from a theoretical ecologist or parasitologist review.
We are happy to recommend more researchers if that helps.

With kind regards on behalf of the authors,\\
\\
Phuong L. Nguyen


Potential referees for the theory aspect if necessary:

\begin{itemize}
	\item Geoff Parker
\end{itemize}

\end{letter}

\end{document}