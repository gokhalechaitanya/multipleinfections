 \documentclass[10,DIN, pagenumber=false, parskip=half,fromalign=right, fromphone=false,fromemail=true, fromurl=false,fromlogo=true, fromrule=false]{scrlttr2}
\usepackage[mac]{inputenc}
\usepackage{graphicx}
\usepackage{float}
\usepackage{caption}
%\usepackage[ngerman]{babel}
\RequirePackage{graphicx}
\usepackage{hyperref}
\hypersetup{
    colorlinks=true,
    linkcolor=blue,
    filecolor=magenta,      
    urlcolor=blue,
}
\setkomafont{footnote}{\normalcolor \sfseries} 
 \setkomavar{date}{\vspace{-0.05cm}    \sffamily \today}
\setkomavar{fromname}{  \sffamily }
\setkomavar{fromaddress}{ \sffamily  
Dr. Phuong Linh Nguyen \\
Chemin du muse\'e 15\\
University of Fribourg\\
Fribourg, 1700 Switzerland
}
\setkomavar{fromlogo}{\includegraphics[scale=0.53]{Fribourg_Logo}}
\setkomavar{backaddress}{\sffamily Phuong L. Nguyen, Department of Biology, University of Fribourg, Switzerland} 
\setkomavar{fromemail}{\sffamily linh.phuong.nguyen@evobio.eu}

\setkomavar{subject}{Submission to Oikos Journal\\}
\newfloat{figure}{htbp}{figs}

\begin{document}
\sffamily

\begin{letter}{
\sffamily
\vspace{-0.4cm}
Dries Bonte \\
Editor-in-Chief\\
Oikos Journal
}
\opening{\sffamily \vspace{-1cm} Dear Prof. Bonte,}
\vspace{-0.3cm}

Here we submit our manuscript, entitled: ``On multiple infections by parasites with complex life cycles." for consideration for publication as a Research paper in Oikos Journal. 
The draft conforms to the required specifications for the Research papers format.

Parasites with complex life cycles traverse multiple hosts to complete their life cycles.
They have, therefore, evolved various strategies to enable successful transmission, one of which is the famous host-manipulating ability.
Trophically transmitted parasites are well known for modifying the behaviour of their intermediate hosts such that the predation rate by their definitive hosts increases, resulting in higher transmission rates from intermediate to definitive hosts.
Interestingly, such manipulative ability changes when multiple parasites infect a host. 
For instance, predation rates may increase if coinfected parasites are ready to transmit. 
On the other hand, if some parasites are still in their developmental phase and not yet ready to transmit, suppression of the predation rate may be more beneficial.
When interests among coinfected parasites clash, one group may win and dominate the manipulative outcomes.

In this manuscript, we use a mathematical model to study the effect of host manipulation on the ecological dynamics of a trophically transmitted parasite.
Specifically, we focus on the presence of multiple parasites of the same pathology, a concept studied thoroughly in infectious diseases but not so extensively in parasitology of trophically transmitted diseases.
We also incorporate the parasite's free-living state, making our model more realistic than previous models.
We focus on host manipulation in intermediate hosts, where effective manipulation results in higher predation rates and, thus, higher transmission rates from intermediate hosts to definitive hosts.
Coinfected parasites can cooperate and boost host manipulation,
Nevertheless, they could also sabotage each other and decrease host manipulation.

Our results suggest that host manipulation generally increases the basic reproduction rate $(R_0)$ of the parasite and need not permanently destabilise the predator-prey system, as suggested in other studies.
However, if the reproduction in single infection is not sufficient, the system exhibits bistable dynamics, suggesting that a minor disturbance to the system, for instance, host migration, could cause the parasite to go extinct.
Bistability is guaranteed when manipulation is sabotaged, but under certain circumstances, bistability can be observed when host manipulation and reproduction is coordinated.

Our model shows the importance of considering multiple infections, which is a norm in nature rather than an exception. 
We also show that the free-living state of the parasite may have significant implications on the ecological dynamics.
This novel result helps us understand the dynamics of parasites with complex lifecycles and their multi-host interactions in-depth. 
Our model addresses the gap in theoretical studies of trophically transmitted parasites, and we believe that our results are of interest to the readership of Oikos Journal.

We thank you for taking the time to consider our manuscript, and we look forward to your response.

With kind regards on behalf of the authors,\\
\\
\\
\\
Phuong L. Nguyen

\end{letter}

\end{document}