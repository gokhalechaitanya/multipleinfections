\documentclass{article}
\usepackage{times}
\usepackage[titletoc]{appendix}
\usepackage{graphicx}
\usepackage{lineno}
\usepackage{multirow}
\usepackage[english]{babel}
\usepackage{typearea} 
\usepackage{amssymb}
\usepackage{amsfonts}
\usepackage{amsmath}
\usepackage{enumerate}
\usepackage{mathtools}
\usepackage{graphicx}
\usepackage{wrapfig}
\usepackage{lscape}
\usepackage{rotating}
\newcommand{\Fig}[1]{Figure~\ref{fig:#1}}

\renewcommand{\baselinestretch}{1.5}
\newcommand{\bbar}[1]{\overline{#1}}

\newcommand\scalemath[2]{\scalebox{#1}{\mbox{\ensuremath{\displaystyle #2}}}}
\renewcommand{\familydefault}{\sfdefault}

\usepackage[font={small},labelfont={bf},justification=justified,margin=0.5cm]{caption}

\renewcommand{\thesection}{}
\renewcommand{\thesubsection}{\arabic{section}.\arabic{subsection}}

\usepackage{color}
	 \definecolor{darkred}{rgb}{0.75,0,0}
%	 \definecolor{darkgreen}{rgb}{0,0.5,0}
	 \definecolor{darkblue}{rgb}{0,0,0.75}
%	 \definecolor{magenta}{rgb}{0,0,0.75}
\newcommand{\cha}[1]{\textcolor{darkgreen}{(#1)}}
\newcommand{\maria}[1]{\textcolor{darkred}{(#1)}}


\usepackage{hyperref}
\definecolor{darkgreen}{rgb}{0.1,0.6,0.3}
\definecolor{darkred}{rgb}{0.6,0.3,0.1}
\hypersetup{
    colorlinks=true,       % false: boxed links; true: colored links
    linkcolor=blue,          % color of internal links (change box color with linkbordercolor)
    citecolor=darkgreen,        % color of links to bibliography
    filecolor=magenta,      % color of file links
    urlcolor= black           % color of external links
}

\title{\vspace*{-22mm}\bf Multiple infections and complex life cycles}
%\author{Vaibhvi$^{1}$,
%Chaitanya S. Gokhale$^{2*}$\\
%\normalsize $^1$Molecular Physiology Group, \\
%\normalsize University of Kiel, \\
%\normalsize Am Botanisch Garten 3-9, D-24118 Kiel, Germany.\\
%\normalsize $^2$Research Group for Theoretical Models of Eco-evolutionary Dynamics, \\
%\normalsize Department of Evolutionary Theory, Max Planck Institute for Evolutionary Biology, \\
%\normalsize August-Thienemann-Stra{\ss}e 2, 24306 Pl\"{o}n, Germany.\\
%\normalsize $^{*}$gokhale@evolbio.mpg.de
%}


\date{}

\begin{document}

\linenumbers
\maketitle


\begin{abstract}
Abstract
\end{abstract}


\noindent
Keywords: a,b,c,d


\tableofcontents

\section{Introduction}

Parasites are ubiquitous, many with complex life cycles.
While a complex life cycle can be defined as abrupt ontogenic changes in morphology and/or ecology \cite{Benesh:2016dj}, a complex parasitic life cycle typically involves numerous hosts that a parasite needs to traverse in the process of completing its life cycle.
Helminths are a prime example of how a successive transmission between multiple host species is necessary for developing the next generation.
Thus the worms occupy different niches (hosts) in different stages of their lifecycle, moving through intermediate hosts until reaching a definitive host in which reproduction can finally occur.
Numerous factors determine the transmissibility and infectivity of the parasites along the tropic level of hosts \cite{froelick:PRSB:2021}.
Comparative growth rate, eventual body size, costs benefits of host-manipulation and the probability of eventually finding a mate all play a major role in the development and maintenance of the life-history strategies of such parasites \cha{More references needed}.
An integral part of all the above factors is the comparative approach.
Parasite often co-occur in the same host and this allows for complex within-host interactions since the evolutionary effect of the actions is not realised all the way till reproduction occurs in the definitive host \cite{Hafer:2015gl}.
In this manuscript we focus on dissecting the fundamental trade-off between transmissibility and host-manipulation when multiple hosts are present in a host and its evolutionary outcome in the associated trait space.


\section{Model}

Trying to list of the main questions that we aim to tackle
\subsection{Parasite life cycles}
\begin{itemize}
	\item Parasites live in hosts
	\item Multiple parasites can live in the same host
	\item Focus on parasites with complex life cycles which need to reproduce in the ``final" host.
	\item Effect of intra-host interaction of multiple parasites affecting transmission into final host. 
\end{itemize}

Decide on traits of interest! - manipulation (transmission)/fecundity tradeoff
Is manipulation cooperative? spiteful?

\subsection{Parasite dynamics only - infinite vectors/hosts}

We assume a large population for hosts and vectors to simplify the model and focus on the effect of multiple parasites on the trait evolution

We look at the evolution of manipulation of a parasite with complex lifecycle, that is, the parasite requires two hosts to complete its lifecycle. Parasites with complex lifecycle enter the intermediate host (vector) from external environment (parasite pool), then the infected intermediate host got eaten by the final host (host), finally the parasite reproduce inside the final host and its offspring is released into the parasite pool.

If we only take into account the dynamics of the parasites, and consider the populations of the vector and the host as infinite and do not have any feedback on the parasite population, then the fitness of a mutant parasite will be:

\textbf{Parasite fitness = Transmission from parasite pool to the vector x Transmission from vector to host x Reproduction in host}

For simplicity, we ignore the competition among parasites in the pool, and ignore that the success of the transmission from the pool to the vector depends on the number of parasite in the vector, then the transmission from the parasite pool to the vector can be consider a fix parameter (let's call it $\tau$).

The transmission from vector to host depends on the strategy of the mutant and the strategy of the resident parasites that coexist with it inside the vector. Let's consider the case when co-infection augment the manipulative ability of all parasites, thus, it augment the transmission rate of all parasite (In reality, coinfection can sabotage the transmission). In this case, we assume that the transmission is a sigmoid function of the net manipulation strategy of all parasite inside the vector

$$\beta(x_m, x_r) = \frac{\beta_0}{1 + e^{-(x_m + x_r k)}}$$

where $k$ is the number of the resident parasites that cohabit with the mutant parasite. We assume that the parasite distribution inside vector follow a negative binomial distribution. Thus, for different number of parasites inside the vector, we will obtain a different value of transmission from vector to host, and so the final transmission value will be

\begin{align*}
    \beta_{total}(x_m) = \sum_{k = 0}^{k = kmax} \frac{\beta_0}{1 + e^{-(x_r + x_m  k)}} \mathcal{BN}(r, k, p)
\end{align*}

where 

$$\mathcal{BN}(r, k, p)
= \begin{pmatrix} k + r - 1 \\ k\end{pmatrix}p^r (1 -p)^k$$

is the negative binomial distribution with $r$ the coefficient that describes the skew of the distribution ($r \rightarrow \inf$ means that the distribution of the parasites follow a normal distribution). $k$ is the number of resident parasites inside the vector, and $p$ is the prevalence of the parasites inside the vector. $kmax$ is the maximum number of neighbour parasites that can coexist with the focal parasite.

The reproduction inside the final host is negatively correlated with the investment in manipulation. Thus the more the parasite invest in manipulation at the vector stage, the lesser investment it will spend on reproduction 

\begin{align*}
    \rho(x) = \frac{\rho_{max}}{1 + a x}
\end{align*}

where $a$ determine the concavity of the function. The more concave the function, the faster the parasite will lose the reproduction value as it invests more in manipulation. If the parasite does not have to invest in manipulation, it will have a maximum $\rho_{max}$. As it invest more in vector manipulation, its reproduction will diminish to 0.

With all the reasoning above, the fitness of a mutant parasite becomes
\begin{align*}
    W(x_m) = \tau \beta_{total}(x_m) r(x_m) = \tau \sum_{k = 0}^{k = kmax} \frac{\beta_0}{1 + e^{-(x_m + x_r  k)}} \begin{pmatrix} k + r - 1 \\ k\end{pmatrix}p^r (1 -p)^k  \frac{\rho_{max}}{1 + a x_m}
\end{align*}
This model does not take into account the order in which the mutant enter the vector and the final host compare to the resident. There is also no ecological feedback of the host and vector dynamics on the selective force on the manipulation.

\textbf{Some thoughts:} 
- We can play around with parameter $r$ that change the skewness of the negative binomial distribution. So what will change if the majority of vectors are infected with only 1 parasites compared with if the majority of vectors are infected with 3 - 4 parasites.

- It's still not clear to me what's the meaning if we take into account the order of entering the host and vector of the mutant. I think that it will require a very careful analysis of the SI model with two strains of parasites. 

- And what if we analyse this model with $k = 2$ and compare with the other model with ecological feedback?

- Here I did not take into account the relatedness between mutants and resident. 

- We can also let the reproduction inside the host affected by the density of the parasites inside the host. But I'm not sure if it would be useful in this case when there is no ecological feedback. Still thinking about this.

\subsection{Host - Vector - Parasite dynamics ?}

The final hosts eat the vectors (subject to manipulations the harvest rates are different).
Including this predator prey population dynamics on the parasite trait evolution.


If we want to take into account the dynamics of the hosts and vectors, we need to limit the number of parasites inside hosts and vectors into two.

Assuming that the total number of hosts and vector are fixed $N_h = I_1 + I_2 + I_{12} + S_h$ and $N_v = V_1 + V_2 + V_{12} + S_v$, the ecological dynamics of hosts, vectors and parasites are as followed

\begin{align*}
&  \frac{dV_1}{dt} = \gamma_1 S_v I_1  +  \gamma_{12} S_v I_{12}  - \gamma_2 V_1 I_2 - d_v V_1 - \alpha_1 V_1 \\
& \frac{dV_2}{dt} = \gamma_2 S_v I_2 + \gamma_{21} S_v I_{12} - \gamma_1 V_2 I_1 -   d_v V_2 - \alpha_2 V_2 \\
& \frac{dV_{12}}{dt} = \gamma_1 V_2 I_1 + \gamma_2 V_1 I_2  -   d_v V_{12} - \alpha_{12} V_{12}\\
& \frac{dI_1}{dt} = \beta_1 S_h V_1  + \beta_{12} S_h V_{12}  - \beta_{21}  I_1 V_{12} - \sigma_1 I_1  - d_h I_1\\
& \frac{dI_2}{dt} = \beta_2 S_h V_2 + \beta_{21} S_h V_{12} - \beta_{12} I_2 V_{12} - \sigma_2 I_2 - d_h I_2\\
& \frac{dI_{12}}{dt} = \beta_2 I_1 V_2 + \beta_1 I_2 V_1 + \beta_{21} I_1 V_{12} + \beta_{12} I_2 V_{12} - \sigma_{12} I_{12} - d_h I_{12}
\end{align*}

In this model, the parasites do not experience external environment, i.e. they do not enter the parasite pool. However, in the model without the dynamics of the hosts and the vectors, we do not consider competition among parasites in the parasite pool. Therefore, the whole term $\rho(x) \tau$ represents the transmission from host to vector. The model with coinfection of two strains of parasites has six compartments \\ description of parameters and variables are in Table \ref{table:varpardescription}.

\begin{table}
\begin{tabular}{|p{3cm}|p{8cm}|} 
\hline
Parameters and Variables    &  Description  \\
\hline
$V_i$     & Vector infected by parasite strain i ($i = 1, 2$). Vector is infected by both strains with $i = 12$ \\
\hline
$I_i$ & Final host infected by parasite strain i ($i = 1, 2$). Final host is infected by both strains with $i = 12$ \\
\hline
$S_v$ & Susceptible vector \\
\hline
$S_h$ & Susceptible final host \\
\hline
$\gamma_i$ & Transmission rate of from hosts infected by parasite strain $i$ to vector. This can be understood as the reproduction rate of parasite $i$ inside the singly infected final host ($i = 1, 2$) \\
\hline
$\gamma_{12}$ & Transmission rate of parasite 1 from hosts co-infected by both strains to vectors (This can be understood as the reproduction rate of parasite 1 inside the doubly infected final host) \\
\hline
$\gamma_{21}$ & Transmission rate of parasite 2 from hosts co-infected by both strains to vectors (This can be understood as the reproduction rate of parasite 2 inside the doubly infected final host) \\
\hline
$d_h$ & Natural death rate of the hosts \\
\hline
$\sigma_i$ & Additional death rate of the hosts due to being infected by parasite i ($i = 1, 2, 12$) \\
\hline
$\beta_i$ & Transmission rate from vectors infected by parasites strain i to hosts. \\
\hline
$\beta_{12}$ & Transmission rate of parasite 1 from vectors coinfected by both strains to hosts.\\
\hline
$\beta_{21}$ & Transmission rate of parasite 2 from vectors coinfected by both strains to hosts. \\
\hline
$\alpha_i$ & Additional death rate of the vectors due to being infected by parasite i ($i = 1, 2, 12$). \\
\hline
\end{tabular}
\caption{Description of variables and parameters}
\label{table:varpardescription}
\end{table}

Here co-infection of both parasites happens consecutively instead of both parasites get into the vectors and the hosts at the same time. Therefore, there are populations of singly infected vectors and hosts that exist for sometime before being infected by the second strains. 

\subsubsection{The dynamics of the mutant}

Considering a mutant of parasite strain 1 that enter the resident population at equilibrium, its dynamics will be

\begin{align*}
& \frac{dV_{1m}}{dt} = \gamma_{1m}  \mathcal{V} I_{1m} + \gamma_{12m} \mathcal{V} I_{12m} - \gamma_2 \hat{I}_2 V_{1m} - \alpha_1 V_{1m}  - d_v V_{1m} \\
& \frac{dV_{12m}}{dt} = \gamma_{1m} \hat{V}_2 I_{1m} +  \gamma_{12m} \hat{V}_2 I_{12m} + \gamma_2 V_{1m} \hat{I}_2 \
- \alpha_{12} V_{12m} - d_v V_{12m} \\
& \frac{dI_{1m}}{dt} =  \beta_{1m} \mathcal{S} V_{1m}  + \beta_{12m} \mathcal{S} V_{12m}  - \beta_{21} I_{1m} \hat{V}_{12} - \sigma_1 I_{1m}  - d_h I_{1m} \\
& \frac{dI_{12m}}{dt} = \beta_2 I_{1m} \hat{V}_2 +  \beta_{1m}  \hat{I}_2 V_{1m} + \beta_{21} I_{1m} \hat{V}_{12}  + 
\beta_{12m} \hat{I}_2 V_{12m}  - \sigma_12 I_{12m} - d_h I_{12m} \\
\end{align*}

We do not allow coinfection of mutant strain with its resident, hence there is no $V_{11m}$ and $I_{11m}$ class. The hats indicate the infected resident population at equilibrium. $\mathcal{V} = N_v - \hat{V}_1 - \hat{V}_2 - \hat{V}_{12}$ and $\mathcal{S} = N_h -\hat{I}_1 - \hat{I}_2 - \hat{I}_{12}$ are respectively the susceptible vectors and final hosts at the resident equilibrium. The mutant is different from the resident in terms of the transmission from host to vector $\gamma_{1m}, \gamma_{12m}$, and the transmission from vector to host $\beta_{1m}, \beta_{12m}$.

\subsection{Mutant's fitness}
The invasion fitness of the mutant will be calculated using the next generation method (Hurford et al 2010).

The matrix $\textbf{A}$ that describes the dynamics of the mutant can be rewritten as $\textbf{A} = \textbf{F} - \textbf{V}$, where $\textbf{F}$ is the matrix that describe the contribution of a compartment to the next generation. \begin{align*}
    \mathbf{F} = 
    \begin{pmatrix}
    0 & 0 & \mathcal{V} \gamma_{1m} & \mathcal{V} \gamma_{12m} \\
    0 & 0 & \gamma_{1m} \hat{V}_2 & \gamma_{12m} \hat{V}_2 \\
    0 & 0 & 0 & 0 \\
    0 & 0 & 0 & 0
    \end{pmatrix}
\end{align*}
Here, because the parasites have a complex lifecycle, where the parasites first enter the vector, then transmit to the host and complete its lifecycle, therefore, a new generation begins with the transmission of the parent parasite from the host to the vector. Hence, contribution to the next generation only involves the final host class $I_{1m}, I_{12m}$. The matrix $\mathbf{V}$ describes the death rates of all compartments and transitions from one compartment to the others
\begin{align*}
    \mathbf{V} = 
    \begin{pmatrix}
     \alpha_1 + \gamma_2 \hat{I}_2 + d_v & 0 & 0 & 0 \\
     -\gamma_2 \hat{I}_2 &  \alpha_{12} + d_v &  0 &  0 \\ -\mathcal{S} \beta_{1m} & -\mathcal{S} \beta_{12m} & \sigma_1 + \beta_{12} \hat{V}_{12} + d_h &   0 \\ -\beta_{1m} \hat{I}_2 & -\beta_{12m} \hat{I}_2 & -\beta_2 \hat{V}_2 - \beta_{12} \hat{V}_{12} & \sigma_{12} + d_h
    \end{pmatrix}
\end{align*}
According to the next generation method, the reproduction value $R_0$ of the mutant is the spectral radius of the matrix $\mathbf{F.V^{-1}}$; here it is the eigenvalue of the matrix, which is
\begin{subequations}
\begin{align}
R_0 = 
    & \frac{\mathit{q}_{HV}}{\mathit{m} + \hat{I}_2 \gamma_2} \frac{\mathit{q}_{VH}}{\mathit{d} + \hat{V}_{12} \beta_{21}} + \label{R0:singlyinfecthv} \\
    & \frac{\ell_{HV}}{\mathit{m} + \hat{I}_2 \gamma_2} \frac{\mathcal{L}_{VH}}{\mathcal{D}} + \frac{\ell_{HV}}{\mathit{m} + \hat{I}_2 \gamma_2} \frac{\mathit{q}_{VH}}{\mathit{d} + \hat{V}_{12} \beta_{21}} \frac{\hat{V}_2 \beta_2 + \hat{V}_{12} \beta_{12}}{\mathcal{D}} + \label{R0:singlyinfectv} \\
    & \frac{\mathcal{Q}_{HV}}{\mathcal{M}} \frac{\ell_{VH}}{\mathit{d} + \hat{V}_{12} \beta_{21}} + \frac{\mathit{q}_{HV}}{\mathit{m} + \hat{I}_2 \gamma_2} \frac{\hat{I}_2 \gamma_2}{\mathcal{M}} \frac{\ell_{VH}}{\mathit{d} + \hat{V}_{12} \beta_{12}} + \label{R0:singlyinfecth} \\
    & \frac{\mathcal{L}_{HV}}{\mathcal{M}} \frac{\mathcal{Q}_{VH}}{\mathcal{D}} + \frac{\ell_{HV}}{\mathit{m} + \hat{I}_2 \gamma_2} \frac{\hat{I}_2 \gamma_2}{\mathcal{M}} \frac{\mathcal{Q}_{VH}}{\mathcal{D}} + \frac{\mathcal{L}_{HV}}{\mathcal{M}} \frac{\ell_{VH}}{\mathit{d} + \hat{V}_{12} \beta_{21}} \frac{\hat{V}_2 \beta_2 + \hat{V}_{12} \beta_{21}}{\mathcal{D}} \label{R0:doublyinfect} \\
    & \frac{\ell_{HV}}{\mathit{m} + \hat{I}_2 \gamma_2} \frac{\hat{I}_2 \gamma_2}{\mathcal{M}} \left( \frac{\ell_{VH}}{\mathit{d} + \hat{V}_{12} \beta_{12}} \frac{\hat{V}_2 \beta_2}{\mathcal{D}} + \frac{\ell_{VH}}{\mathit{d} + \hat{V}_{12} \beta_{21}} \frac{\hat{V}_{12} \beta_{21}}{\mathcal{M}} \right) \notag
\end{align}
\end{subequations}
where $\mathcal{Q}_i, \mathcal{L}_i, \mathit{q}_i$, and $\ell_i$ represents the force of infection. Note that here there are two types of forces of infection, from host to vector $i = HV$ and from vector to host $i = VH$, because the parasite has a complex lifecycle. $\mathcal{D}, \mathcal{M}, \mathit{d}$, and $\mathit{m}$ are different death rates. Moreover, because we take into account multiple infection, the force of infection does not only involve susceptible individuals as it is usually defined in the standard epidemiology model. Details explanations can be found in Table \ref{table:R0annotation}. 
\begin{table}[!h]
    \centering
    \begin{tabular}{|c|l|}
    \hline
        Symbols &  Description \\
    \hline
    \multicolumn{2}{|l|}{Force of infection from host to vector} \\
    \hline
       $\mathcal{Q}_{HV} = \hat{V}_2 \gamma_{1m}$ & host singly infected by mutant to vector singly infected by strain 2 \\
    \hline
       $ \mathit{q}_{HV} = \mathcal{V} \gamma_{1m}$ & host singly infected by mutant to susceptible vector \\
    \hline
       $\mathcal{L}_{HV} = \hat{V}_2 \gamma_{12m}$ & host doubly infected to vector singly infected by strain 2 \\
    \hline
       $\ell_{HV} = \mathcal{V} \gamma_{12m}$ & host doubly infected to susceptible vector \\
    \hline
    \multicolumn{2}{|l|}{Force of infection from vector to host} \\
    \hline
       $\mathcal{Q}_{VH} = \hat{I}_2 \beta_{12m}$ & vector doubly infected to host singly infected by strain 2 \\
    \hline
       $\mathit{q}_{VH} = \mathcal{S} \beta_{1m}$ &  vector singly infected by mutant to susceptible host \\
     \hline
       $\mathcal{L}_{VH} = \hat{I}_2 \beta_{1m}$ & vector singly infected by mutant to host singly infected by strain 2 \\
    \hline
       $\ell_{VH} = \mathcal{S} \beta_{12m}$ &  vector doubly infected to susceptible host \\
    \hline
    \multicolumn{2}{|l|}{Total death rate} \\
       $\mathcal{D} = d_h + \sigma_{12}$ & Total death of the final host that is doubly infected \\
     \hline
       $\mathit{d} = d_h + \sigma_1$ & Total death of the final host that is singly infected by mutant \\
    \hline
       $\mathcal{M} = d_v + \alpha_{12}$ & Total death of the vector that is doubly infected \\
    \hline
       $\mathit{m} = d_v + \alpha_1$ & Total death of the vector that is singly infected by mutant \\
    \hline
    \end{tabular}
    \caption{Explanation of terms in $R_0$}
    \label{table:R0annotation}
\end{table}The reproduction ratio of the mutant is composed of four compartments which represents four possibilities in which the mutant find itself: (\ref{R0:singlyinfecthv}) the mutant is alone in both vector and host, (\ref{R0:singlyinfectv}) the mutant is alone in the vector but cohabits with strain 2 in the host, (\ref{R0:singlyinfecth}) the mutant cohabits with strain 2 in the vector but is alone in the host, and (\ref{R0:doublyinfect}) the mutant cohabits with strain 2 in both vector and host. Moreover, in multiple infection scenarios, mutant can be the first that infects vectors and hosts but it can also come after an individual of strain 2 has already occupied the vectors and the hosts.
\section{Discussion}



\textbf{Code availability}.
Appropriate {\tt{xyz}} computer code describing the model is available at {\url{https://github.com/tecoevo/xyz}}.



\bibliographystyle{naturemag}
\bibliography{references.bib}


\end{document}