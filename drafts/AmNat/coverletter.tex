Dear Editor,

We would like to submit our manuscript, entitled: "On multiple infections by parasites with complex life cycles"
to the American Naturalist.

Parasites with complex life cycles need to traverse at least two hosts to complete their life cycles.
They therefore evolve various trategies to enable successful transmissions, 
one of which is the famous host manipulating ability.
Trophically transmitted parasites are wellknown for being able to modify the behaviour of their
intermediate hosts such that the predation rate by their definitive hosts increases, 
resulting in higher transmission rates from intermediate to definitive hosts.
Interestingly, such manipulative ability changes when multiple parasites infect a host. 
For instance, predation rates may increase if coinfected parasites are ready to transmit. 
On the other hand, if some parasites are still in their developmental phase and not yet
ready to transmit, suppression of predation rate may be more beneficial.
When interests among coinfected parasites clashes, one group may win and dominate the manipulative outcomes.

In this manuscript, we use mathematical model to study the effect of host manipulation 
on the ecological dynamics of a trophically transmitted parasite.
We introduce multiple infections into the system, which has been studied thoroughly in
infectious diseases but not so extensively in trophically transmitted parasites.
We also incorporate the free-living state of the parasite, which makes our model more realistic than previous models.
We focus on host manipulation in intermediate hosts, where effective manipulation results in higher predation rates, 
and thus, higher transmission rate from intermediate hosts to definitive hosts.
Coinfected parasites can cooperate and boost host manipulation,
but they could also sabotage each other and decrease host manipulation.

Our results suggest that host manipulation in general increases the basic reproduction rate $(R_0)$ of the parasite,
and need not always destabilise the predator-prey system as suggested in other studies.
However, cooperation that increases host manipulation leads to bistability of the system if parasite reproduction is sufficiently large.
This suggests that a small disturbance of the system, for instance, hosts migration, could cause the parasite to go extinct.
Cooperation in both host manipulation and reproduction may do more harm than good, by broadening the area of bistability.
Finally, when parasites are uncooperative, stability of the system is always guaranteed.

Our model shows the importance of considering multiple infections, which is a norm rather than an exception in nature. 
We also show that the free-living state of the parasite may have important implications on the ecological dynamics.
This, to our best knowledge, has never been taken into account in any model. 
Our model addresses the gap in theoretical studies of trophically transmitted parasites,
and we believe that our results are of interest to the broad audience of the American Naturalist.

We thank the editor for taking our manuscript into consideration, and we look forward to your response.

Phuong Linh Nguyen and Chaitanya Gokhale

