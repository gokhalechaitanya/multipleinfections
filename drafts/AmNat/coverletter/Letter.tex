 \documentclass[10,DIN, pagenumber=false, parskip=half,fromalign=right, fromphone=false,fromemail=false, fromurl=false,fromlogo=true, fromrule=false]{scrlttr2}
\usepackage[mac]{inputenc}
\usepackage{graphicx}
\usepackage{float}
\usepackage{caption}
\usepackage[ngerman]{babel}
\RequirePackage{graphicx}
\usepackage{hyperref}
\hypersetup{
    colorlinks=true,
    linkcolor=blue,
    filecolor=magenta,      
    urlcolor=blue,
}
\setkomafont{footnote}{\normalcolor \sfseries} 
 \setkomavar{date}{\vspace{-0.05cm}    \sffamily \today}
\setkomavar{fromname}{  \sffamily Prof. Dr. Chaitanya S. Gokhale }
\setkomavar{fromaddress}{ \sffamily  
CCTB\\
Klara-Oppenheimer Weg 32\\
Julius-Maximilians University \\
%August Thienemann Str-2, 24306\\
W\"{u}rzburg, 97070 Germany
}
\setkomavar{fromlogo}{\includegraphics[scale=0.33]{CCTB_Logo}}
\setkomavar{backaddress}{\sffamily Chaitanya S. Gokhale, Center for Computational and Theoretical Biology, Germany} 
\setkomavar{fromemail}{\sffamily lauenroth@evolbio.mpg.de}

\setkomavar{subject}{Submission to The American Naturalist\\}
\newfloat{figure}{htbp}{figs}

\begin{document}
\sffamily

\begin{letter}{
\sffamily
\vspace{-0.4cm}
Volker H. W. Rudolf,\\
Editor-in-Chief\\
The American Naturalist
}
\opening{\sffamily \vspace{-1cm} Dear Dr. Rudolf,}
\vspace{-0.3cm}

Here we submit our manuscript, entitled: ``On multiple infections by parasites with complex life cycles." for consideration for publication in the American Naturalist.

Parasites with complex life cycles traverse multiple hosts to complete their life cycles.
They have therefore evolved various strategies to enable successful transmission, one of which is the famous host-manipulating ability.
Trophically transmitted parasites are well known for modifying the behaviour of their intermediate hosts such that the predation rate by their definitive hosts increases, resulting in higher transmission rates from intermediate to definitive hosts.
Interestingly, such manipulative ability changes when multiple parasites infect a host. 
For instance, predation rates may increase if coinfected parasites are ready to transmit. 
On the other hand, if some parasites are still in their developmental phase and not yet
ready to transmit, suppression of the predation rate may be more beneficial.
When interests among coinfected parasites clash, one group may win and dominate the manipulative outcomes.

In this manuscript, we use a mathematical model to study the effect of host manipulation on the ecological dynamics of a trophically transmitted parasite.
We introduce multiple infections into the system, which has been studied thoroughly in
infectious diseases but not so extensively in trophically transmitted parasites.
We also incorporate the parasite's free-living state, making our model more realistic than previous models.
We focus on host manipulation in intermediate hosts, where effective manipulation results in higher predation rates and, thus, higher transmission rate from intermediate hosts to definitive hosts.
Coinfected parasites can cooperate and boost host manipulation,
Nevertheless, they could also sabotage each other and decrease host manipulation.

Our results suggest that host manipulation generally increases the basic reproduction rate $(R_0)$ of the parasite and need not permanently destabilise the predator-prey system, as suggested in other studies.
However, cooperation that increases host manipulation leads to a bistable system if parasite reproduction is sufficiently large.
This result suggests that a minor disturbance to the system, for instance, host migration, could cause the parasite to go extinct.
Cooperation in both host manipulation and reproduction may do more harm than good by broadening the area of bistability.
Finally, the system's stability is always guaranteed when parasites are uncooperative.

Our model shows the importance of considering multiple infections, which is a norm in nature rather than an exception. 
We also show that the free-living state of the parasite may have significant implications on the ecological dynamics.
To our knowledge, this result has not been considered in any theoretical studies on host-parasite interactions. 
Our model addresses the gap in theoretical studies of trophically transmitted parasites, and we believe that our results are of interest to the broad audience of the American Naturalist.

We thank you for taking the time to consider our manuscript, and we look forward to your response.

With kind regards on behalf of the authors,\\
\\
\\
\\
Chaitanya S. Gokhale

\end{letter}

\end{document}