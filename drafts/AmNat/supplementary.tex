\documentclass[11pt]{article}
\usepackage[sc]{mathpazo} %Like Palatino with extensive math support
\usepackage{fullpage}
\usepackage[authoryear,sectionbib,sort]{natbib}
\linespread{1.7}
\usepackage[utf8]{inputenc}
\usepackage{lineno}
\usepackage{titlesec}
\usepackage{graphicx}
\usepackage{color}
\usepackage{amsmath}  % for equations

\title{Supplementary Information:\\ On multiple infections by parasites with complex life cycles}

% This version of the LaTeX template was last updated on
% November 8, 2019.

%%%%%%%%%%%%%%%%%%%%%
% Authorship
%%%%%%%%%%%%%%%%%%%%%
% Please remove authorship information while your paper is under review,
% unless you wish to waive your anonymity under double-blind review. You
% will need to add this information back in to your final files after
% acceptance.

%\author{Phuong Linh Nguyen$^{1,\ast}$ \& 
%Chaitanya S. Gokhale$^{2,3}$
%}
\date{}

\begin{document}
\maketitle

\section*{SI 1. Reproduction ratio $R_0$}

The reproduction ratio of the parasite is derived from the dynamical system of the parasite which only include infected intermediate and definitive hosts and the free-living parasite pool. The dynamical system can be written in matrix form as followed:
\[
\frac{d \mathbf{n}}{dt} = \mathbf{M} \mathbf{n}
\]

where $\mathbf{n}$ is the vector of singly and doubly infected intermediate hosts, singly and doubly infected definitive hosts and free-living parasites ($dI_w, I_{ww}, D_w, D_{ww}, W$) and $\mathbf{M}$ is the matrix that describes the dynamics
\[ \mathbf{M} = 
\begin{pmatrix}
- d - \alpha_w - P_w & 0 & 0 & 0 & (1 - p) \gamma I_s \\
0 & -d - \alpha_{ww} - P_{ww} & 0 & 0 & p \gamma I_s \\
\beta_w D_s & 2 \beta_{ww} (1-q) D_s &-\lambda_w - 2  (1-q)\lambda_{ww}  -\mu -\sigma_w & 0 & 0 \\
0 & \beta_{ww} q D_s  & \lambda_w + 2 (1-q) \lambda_{ww}  & -\mu - \sigma_{ww} & 0 \\
 0 & 0 & f_w & f_{ww} &- \delta - \gamma I_s  \\
\end{pmatrix}
\]

The matrix $\mathbf{M}$ can be written as $\mathbf{M} = \mathbf{F} - \mathbf{V}$, where

\[
\mathbf{F} = 
\begin{pmatrix}
0 & 0 & 0 & 0 & 0  \\
0 & 0 & 0 & 0 & 0  \\
0 & 0 & 0 & 0 & 0  \\
0 & 0 & 0 & 0 & 0  \\
0 & 0 & f_w & f_{ww} & 0
\end{pmatrix}
\]

is the matrix in which its elements are the reproduction contribution of one compartment to the other compartments in the next generation, and

\[
\mathbf{V} = 
\begin{pmatrix}
\alpha_w + d + P_w  & 0 & 0 & 0 & - (1 - p) \gamma  I_s\\
 0 & \alpha_{ww} + d + P_{ww} & 0 & 0 & - p \gamma I_s \\
- \beta_w D_s & -2 \beta_{ww}  (1 - q) D_s & \lambda_w + 2 \lambda_{ww} (1-q) + \mu + \sigma_w & 0 & 0 \\
 0 & -\beta_{ww} q D_s  & -\lambda_w - 2 \lambda_{ww} (1 - q) & \mu + \sigma_{ww} & 0 \\
 0 & 0 & 0 & 0 & \delta + \gamma I_s \\
\end{pmatrix}
\]

is the matrix in which its elements include death rates or transition rates from one compartment to the others. \textcolor{red}{(Whom should we cite for the next generation method?)}

The reproduction ratio $R_0$ is then the leading eigenvalue of the matrix $\mathbf{F}.\mathbf{V}^-1$, evaluated at the disease-free equilibrium of the intermediate and definitive hosts $I_s^*$, $D_s^*$, and $I_w = I_{ww} = D_w = D_{ww} = 0$.

\section*{SI 2. Linear birth function for intermediate hosts}

\subsection*{Equilibrium stability}

The jacobian matrix of the system (1), (2), (3), \textcolor{red}{(We need to be a bit more explicit here when connecting ot the main text)} evaluated at the disease-free equilibrium, and $B(D_s, D_w, D_{ww}, I_s, I_w, I_{ww}) = \rho c D_{total} I_{total}$ is

\[
\begin{pmatrix}
0 & r & r & -\frac{\mu}{c} & -\frac{\mu }{c} & -\frac{\mu }{c} & -\frac{\gamma  \mu }{c \rho } \\
0 & -\alpha_w + \frac{(\beta_w + \rho ) (d-r)}{\rho } - d & 0 & 0 & 0 & 0 & \frac{\gamma  \mu  (1-p)}{c \rho } \\
0 & 0 & -\alpha_{ww} + \frac{(\beta_{ww} + \rho ) (d-r)}{\rho } - d & 0 & 0 & 0 & \frac{\gamma  \mu  p}{c \rho } \\
-c (d-r) & \frac{\beta_w (d-r)}{\rho } - c (d-r) & \frac{\beta_{ww} (d-r)}{\rho }-c (d-r) & 0 & \mu  & \mu  & 0 \\
0 & -\frac{\beta_w (d-r)}{\rho } & -\frac{2 \beta_{ww} (1-q) (d-r)}{\rho } & 0 & -\mu - \sigma_w & 0 & 0 \\
 0 & 0 & -\frac{\beta_{ww} q (d-r)}{\rho } & 0 & 0 & -\mu - \sigma_{ww} & 0 \\
 0 & 0 & 0 & 0 & f_w & f_{ww} & -\frac{\gamma  \mu }{c \rho } - \delta  \\
\end{pmatrix}
\]

This jacobian has seven eigenvalues, two of which have explicit expressions as $\pm \sqrt{d - r}$. Here, we always have $r > d$ so that the equilibrium is positive, therefore these two eigenvalues are always pure imaginary. We cannot obtain the explicit expression of the other five eigenvalues but the dynamics remain unstable regardless of their values.

\subsection*{Invasion of parasites}

$R_0 > 1$ when the transmission rate from the parasite pool to intermediate hosts satisfies

\begin{align}
	\gamma > & \frac{c \delta  \rho  (\mu + \sigma_w) (\mu + \sigma_{ww}) (-\alpha_w \rho + \beta_w d-r (\beta_w+\rho )) (\alpha_{ww} \rho + \beta_{ww} (-d)+r (\beta_{ww}+\rho ))}{\mu } \notag \\
	& \frac{1}{\begin{pmatrix}
		\beta_w \beta_{ww} d^2 (f_w (\mu +\sigma_{ww}) (2 p q-p-1) + (\mu +\sigma_w) (-f_{ww} p q+\mu +\sigma_{ww})) - & \\
		 d (f_w (\mu +\sigma_{ww}) (2 \beta_{ww} p (q-1) \rho  (\alpha_w+r)+2 \beta_w \beta_{ww} r (p (2 q-1)-1)+ & \\
		 \beta_w (p-1) \rho  (\alpha_{ww}+r)) + (\mu +\sigma_w) ((\mu +\sigma_{ww}) (\beta_{ww} \rho  (\alpha_w+r) + & \\ 
		 \beta_w \rho  (\alpha_{ww}+r)+2 \beta_w \beta_{ww} r)-\beta_{ww} f_{ww} p q (\rho  (\alpha_w+r)+2 \beta_w r))) + & \\
		 f_w r (\mu +\sigma_{ww}) (2 \beta_{ww} p (q-1) \rho  (\alpha_w+r) +  & \\
		 \beta_w \beta_{ww} r (p (2 q-1)-1) + \beta_w (p-1) \rho  (\alpha_{ww}+r)) + & \\
		 (\mu +\sigma_w) (\alpha_w \rho +r (\beta_w+\rho )) ((\mu +\sigma_{ww}) (\rho  (\alpha_{ww}+r)+\beta_{ww} r)-\beta_{ww} f_{ww} p q r)
		\end{pmatrix}}
\end{align}

and the reproduction rates $f_w$ and $f_{ww}$ satisfies either of the following conditions

\begin{align}
f_{ww}\geq \frac{-\alpha_{ww} \mu  \rho -\alpha_{ww} \rho  \sigma_{ww} + \beta_{ww} d \mu +\beta_{ww} d \sigma_{ww}-\beta_{ww} \mu  r-\beta_{ww} r \sigma_{ww}-\mu  \rho  r-\rho  r \sigma_{ww}}{\beta_{ww} d p q-\beta_{ww} p q r}
\end{align}

or

\begin{align}
	f_{ww} < & \frac{-\alpha_{ww} \mu  \rho -\alpha_{ww} \rho  \sigma_{ww} + \beta_{ww} d \mu +\beta_{ww} d \sigma_{ww}-\beta_{ww} \mu  r-\beta_{ww} r \sigma_{ww}-\mu  \rho  r-\rho  r \sigma_{ww}}{\beta_{ww} d p q-\beta_{ww} p q r} \notag \\
	f_w > & \frac{(\mu +\sigma_w) (-\alpha_w \rho + \beta_w d-r (\beta_w+\rho ))}{(d-r) (\mu + \sigma_{ww})} \notag \\
	& \frac{\beta_{ww} d (f_{ww} p q-\mu -\sigma_{ww})+\beta_{ww} (-f_{ww}) p q r+(\mu +\sigma_{ww}) (\rho  (\alpha_{ww}+r)+\beta_{ww} r)}{\beta_w \beta_{ww} (d-r) (p (2 q-1)-1)+\rho  (-2 \beta_{ww} p (q-1) (\alpha_w+r)-(\beta_w (p-1) (\alpha_{ww}+r)))} 
\end{align}


\section*{SI 3. Non-linear birth function for intermediate hosts - invasion condition}

The condition for parasite invasion is $R_0 > 1$, which is satisfied when

\begin{align}
	f_w > & \frac{(c \delta  \rho +\gamma  \mu ) (k \mu  r (\beta_w+\rho )-c \rho  (\beta_w (-d)+\rho  (\alpha +r)+\beta_w r))}{\gamma  \mu  (c \rho  (d-r)+k \mu  r)} \notag \\
	& \frac{(\mu +\sigma ) (k \mu  r (\beta_{ww}+\rho )-c \rho  (\beta_{ww} (-d)+\rho  (\alpha +r)+\beta_{ww} r))}{ \begin{pmatrix} c \rho  (\beta_w \beta_{ww} (d-r) (p q (\epsilon -2)+p+1) + & \\ 
			\rho  (\alpha +r) (\beta_w (p-1)-\beta_{ww} p (q (\epsilon -2)+2))) + & \\
			 k \mu  r (\beta_w (\beta_{ww} (p q (\epsilon -2)+p+1)-p \rho + & \\
			 \rho )+\beta_{ww} p \rho  (q (\epsilon -2)+2)) \end{pmatrix}}
\end{align}
\end{document}