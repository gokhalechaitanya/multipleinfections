\documentclass[11pt]{article}
\usepackage[sc]{mathpazo} %Like Palatino with extensive math support
\usepackage{fullpage}
%\usepackage[authoryear,sectionbib,sort]{natbib}
\linespread{1.7}
\usepackage[utf8]{inputenc}
\usepackage{lineno}
\usepackage{titlesec}
% Added later
\usepackage{amsmath}  % for equations
\usepackage{graphicx} % for figures
\titleformat{\section}[block]{\Large\bfseries\filcenter}{\thesection}{1em}{}
\titleformat{\subsection}[block]{\Large\itshape\filcenter}{\thesubsection}{1em}{}
\titleformat{\subsubsection}[block]{\large\itshape}{\thesubsubsection}{1em}{}
\titleformat{\paragraph}[runin]{\itshape}{\theparagraph}{1em}{}[. ]\renewcommand{\refname}{Literature Cited}

\usepackage{color}
	\definecolor{darkred}{rgb}{0.75,0.1,0.1}
	 \definecolor{darkblue}{rgb}{0,0,0.75}
	 \definecolor{magenta}{rgb}{0,0,0.75}
\newcommand{\cha}[1]{\textcolor{darkred}{(#1)}}

\usepackage{hyperref}
\definecolor{darkgreen}{rgb}{0.1,0.6,0.3}
\hypersetup{
    colorlinks=true,       % false: boxed links; true: colored links
    linkcolor=blue,          % color of internal links (change box color with linkbordercolor)
    citecolor=darkgreen,        % color of links to bibliography
    filecolor=magenta,      % color of file links
    urlcolor= black           % color of external links
}
%%%%%%%%%%%%%%%%%%%%%
% Line numbering
%%%%%%%%%%%%%%%%%%%%%
%
% Please use line numbering with your initial submission and
% subsequent revisions. After acceptance, please turn line numbering
% off by adding percent signs to the lines %\usepackage{lineno} and
% to %\linenumbers{} and %\modulolinenumbers[3] below.
%
% To avoid line numbering being thrown off around math environments,
% the math environments have to be wrapped using
% \begin{linenomath*} and \end{linenomath*}
%
% (Thanks to Vlastimil Krivan for pointing this out to us!)

\title{Multiple infections and complex life cycles}

% This version of the LaTeX template was last updated on
% November 8, 2019.

%%%%%%%%%%%%%%%%%%%%%
% Authorship
%%%%%%%%%%%%%%%%%%%%%
% Please remove authorship information while your paper is under review,
% unless you wish to waive your anonymity under double-blind review. You
% will need to add this information back in to your final files after
% acceptance.

\author{}
\date{}





\begin{document}
%\linenumbers
\maketitle

\noindent{} 1. Research Group for Theoretical Models of Eco-evolutionary Dynamics, Department of Evolutionary Theory, Max Planck Institute for Evolutionary Biology, Pl\"{o}n 24306, Germany;
%
\noindent{} 2. Theoretical Ecology and Evolution Group, Department of Biology, University of Fribourg, Switzerland
%
\noindent{} $\ast$ Corresponding author; e-mail: linh.phuong.nguyen@evobio.eu

\bigskip

\textit{Manuscript elements}:  Figure~1, figure~2, figure~3, figure~4, figure~5, figure~6, % appendices~A and B (including figure~A1, figure~B1 and figure~B2). All figures except figure~3 are in color.

\bigskip

\textit{Keywords}: host manipulation, multiple infections


\bigskip

\textit{Manuscript type}: e-article. %Or e-article, note, e-note, natural history miscellany, e-natural history miscellany, comment, reply, invited symposium, or historical perspective.

\bigskip

% \noindent{\footnotesize Prepared using the suggested \LaTeX{} template for \textit{Am.\ Nat.}}

%\linenumbers{}
%\modulolinenumbers[3]

\newpage{}

\section*{Abstract}



\newpage{}

\section*{Introduction}

% The journal does not have numbered sections in the main portion of
% articles. Please refrain from using section references (à la
% section~\ref{section:CountingOwlEggs}), and refer to sections by name
% (e.g. section ``Counting Owl Eggs'').

Parasites infect life on earth ubiquitously, and many of these parasites have complex life cycles \cite{zimmer:book:2001}. 
While a complex lifecycle can be defined as abrupt ontogenic changes in morphology and ecology \cite{Benesh:2016dj}, a complex parasitic lifecycle typically involves numerous hosts that a parasite needs to traverse to complete its life cycle. 
This complex lifecycle results in the evolution of various strategies that enable the success of parasite transmission from one host to another. 
One famous strategy that inspires many science fiction movies and novels is host manipulation, where a parasite can alter the morphology and/or behaviour of its  host to enhance its transmission to the next host \cite{Hughes2012}. 
Host manipulation has been shown in many host-parasite systems, from parasites with simple life-cycle to those with complex life-cycle that involves more than one host \cite{Hughes2012, molyneux1986}. 
For instance, sand flies infected by \textit{Leishmania} parasites bite more and take more time for a blood meal from mammals (the definitive host of \textit{Leishmania}) compared to their uninfected counterparts \cite{ Rogers2007}. 
Copepods infected by cestode parasites are more active and accessible to sticklebacks (the definitive hosts of the cestodes) compared to uninfected copepods \cite{Wedekind1996}.


Theoretical studies have attempted to understand the ecological and evolutionary consequences of host manipulation. \cite{Roosien2013, Hosack2008} showed that manipulative parasites could increase the disease prevalence in an epidemic. \cite{Gandon2018} studies the evolution of the manipulative ability of infectious disease parasites, showing different evolutionary outcomes depending on whether the pathogen can control its vector or host.
\cite{Hadeler1989, Fenton2006} and \cite{Rogawa2018} showed that host manipulation could stabilise or destabilise the predator-prey dynamics depending on how manipulation affects the predation response function and the assumption on the fertility of the definitive infected host. \cite{Seppl2008} showed that host manipulation could evolve even when it increases the risk of the intermediate host being eaten by a non-host predator, given that the initial predation risk is sufficiently low. 
However, these models do not consider multiple infections, a phenomenon that is the norm rather than an exception in parasitism. Multiple infections result in the coinfection of more than one parasite inside a host, which may vastly alter the manipulative outcomes. 
An alignment of interest between coinfecting parasites may enhance manipulation, while a conflict of interest may reduce the manipulative effect. 
\cite{Hafer:2015gl} showed that copepods infected by two cestode parasites reduce the activity of copepods when both parasites are at the same noninfectious stage, i.e. both parasites are not ready to transmit. Thus the reduction in mobility is suggested to reduce the predation rate by the definitive hosts. When two infectious parasites infect the copepods, the copepods' activity increases, and so does the predation risk. 
However, when the copepods are infected by one infectious and one noninfectious parasite, their interests clash, and one parasite wins over the other. 


Theoretical work that considers multiple infections often focuses on the evolution of virulence \cite{vanBaalen1995, Alizon2013, Alizon2008, Choisy2010, Alizon2012}. 
They show that multiple infections can lead to an increase in virulence \cite{vanBaalen1995, Choisy2010}, a branching of one less virulent and one hypervirulent parasite when within-host dynamics are considered \cite{ Alizon2008}, a reduction in virulence if parasites are co-transmitted \cite{Alizon2012}. 
In epidemiological models, higher virulence is often assumed to link with higher transmission rate, virulence is therefore associated with host manipulation in such cases. 
In trophically transmitted parasites, host manipulation is associated with predation rate, which predominantly affects predator-prey dynamics. 
Theoretical studies on host manipulation in trophically transmitted parasites with multiple infections are rare \cite{Parker2003,Vickery2009}. Moreover, they do not consider the prey-predator dynamics, which could have important feedback on the evolution of host manipulation. 
A few studies that consider the prey-predator dynamics do not incorporate multiple infections \cite{Rogawa2018, Iritani2018, Hadeler1989, Fenton2006}. 
More importantly, they assume that transmission from definitive hosts to intermediate hosts is due to direct contact between the two types of hosts. 
This is often not the case in reality, as parasites are released from the definitive hosts into the environment. 
Transmission happens only when intermediate hosts have contact with this free-living parasite pool.


Our study addresses the gap in the theoretical work on host manipulation in trophically transmitted parasites.
We include multiple infections and consider the dynamics of the free-living parasite pool. 
We use a compartmental model that illustrates a complex lifecycle parasite with two hosts: an intermediate host preyed upon by a definitive host. 
Transmission from the intermediate host to the definitive host occurs when predation on infected intermediate hosts happens. 
Reproduction only happens in the definitive hosts, and new parasites are released into the environment, where they again have contact with the intermediate hosts to complete their lifecycle. 
We focus on the manipulation of the intermediate hosts, such that the parasite increases the predation rate on the intermediate host by the definitive host to increase its transmission rate. 
We analyse the effect of host manipulation on the ecological dynamics of the prey-predator-parasite system, considering manipulation when multiple infections occur. We found that ....

\section*{Model and Results}

We focus on the complex lifecycle of a trophically transmitted parasite that requires two hosts: an intermediate host and a definitive host. 
The parasites only reproduce inside their definitive hosts and their offspring are released in to the environment. An intermediate host can be infected if it encounters this free-living parasite pool. 
Finally, when a definitive host consumes an infected intermediate host, the definitive host gets infected, and the parasite completes its lifecycle.


For simplicity, we assume that hosts can be infected by one (single infection) or, at most, two parasites (double infections). 
Given that infection occurs, the probability that two parasites from the parasite pool co-transmit to an intermediate host is denoted by  $p$, and thus $1-p$ is the probability that a single parasite enters an intermediate host. 
When a definitive host consumes an intermediate host infected by two parasites, there is a probability $q$ that the parasites co-transmit to the definitive host.
With probability $1-q$, only one parasite successfully transmits. 
This formulation assumes that infection always happens when hosts encounter parasites.
The dynamics of a complex lifecycle parasite that requires two hosts is described by the following system of ODEs, firstly for the intermediate host as,
%
\begin{align}
\frac{dI_s}{dt} &= R(I_s, I_w, I_{ww}) - d I_s - P_s(D_s, D_w, D_{ww}) I_s  - \eta  I_s \nonumber \\ 
\frac{dI_w}{dt} &=  (1 - p) \eta I_s  - (d + \alpha_w) I_w - P_w(D_s, D_w, D_{ww}, \beta_w) I_w \label{odes:ihosts} \\
\frac{dI_{ww}}{dt} &= p \eta I_s  - (d + \alpha_{ww}) I_{ww} - P_{ww}(D_s, D_w, D_{ww}, \beta_{ww}) I_{ww} \nonumber
\end{align}
%
where $R(I_s, I_w, I_{ww})$ represents the birth rate of the intermediate hosts, a function of both infected and uninfected individuals.
$P_i$, where $i = \{s, w, ww\}$ is the predation function of definitive hosts on susceptible, singly infected and doubly infected intermediate hosts respectively. 
The predation function depends on the density of the definitive hosts and the manipulative strategies of parasites in the intermediate hosts. 
In particular, if a single parasite infects an intermediate host, the manipulation strategy is $\beta_w$. 
However if the intermediate host is co-infected, the manipulation strategy is $\beta_{ww}$. 
In the scope of this model, we assume no link between $\beta_w$ and $\beta_{ww}$. 
The force of infection by parasites in the environment is denoted by $\eta = \gamma W$. 
The force of infection that corresponds respectively to singly infected intermediate host ($I_w$), or doubly infected intermediate hosts ($I_{ww}$) is denoted respectively by $\lambda_w = \beta_w I_w$ and $\lambda_{ww} = \beta_{ww} I_{ww}$. Because in reality parasites can manipulate both intermediate and definitive hosts, here, whenever we mention host manipulation, it specifically refers to the manipulation in intermediate hosts, which correlates to the predation rate.

For the definitive hosts we have,
\begin{align}
\frac{dD_s}{dt} &= B(D_s,  D_w,  D_{ww},  I_s, I_w, I_{ww})  - \mu D_s - (\lambda_{ww} + \lambda_w) D_s \nonumber \\    
\frac{dD_w}{dt} &= (\lambda_w + 2 (1 - q) \lambda_{ww}) D_s - (\mu + \sigma_w) Dw - (2 (1 - q) \lambda_{ww} + \lambda_w) D_w  \label{odes:dhosts} \\         
\frac{dD_{ww}}{dt} &= q \lambda_{ww} D_s + (2 (1 - q) \lambda_{ww} + \lambda_w) D_w - (\mu + \sigma_{ww}) D_{ww} \nonumber
\end{align}
%
where $B(D_s, D_w, D_{ww}, I_s, I_w, I_{ww})$ represents the birth rate of definitive hosts, which depends on the density of both intermediate and definitive hosts, infected or uninfected alike. 
The dynamics of the free-living parasites in the environment are then given solely by,
\begin{align}
	\frac{dW}{dt} &= f_w D_w + f_{ww} D_{ww} - \delta W - \eta I_s \label{odes:eparasite}
\end{align}


Definitions of different parameters can be found in Table~\ref{table:varpardescription}.
%
\begin{table}[!ht]
\begin{tabular}{|p{2.5cm}|p{12cm}|} 
\hline
Parameters and Variables    &  Description  \\
\hline
$I_i$  & Density of intermediate hosts that are susceptible $i=s$, singly infected $i=w$, or doubly infected $i=ww$ \\
\hline
$D_i$ & Density of definitive hosts that are susceptible $i=s$, singly infected $i=w$, or doubly infected $i=ww$ \\
\hline
$W$ & Density of parasites released from definitive hosts into the environment \\
\hline
$d$ & Natural death rate of intermediate hosts \\
\hline
$\alpha_i$ & Additional death rate of intermediate hosts due to infection by a single parasite ($i = w$) or two parasites ($i = ww$) \\
\hline
$p$ & Probability that two parasites cotransmit from the environment to an intermediate host \\
\hline
$\gamma$ & Transmission rate of parasites in the environment to intermediate hosts \\
\hline
$\mu$ & Natural death rate of definitive hosts \\
\hline
$\sigma_i$ & Additional death rate of definitive hosts due to infection by a single parasite ($i = w$) or two parasites ($i = ww$) \\
\hline
$\sigma_i$ & Additional death rate of the hosts due to being infected by a singly parasite ($i = w$) or two parasites ($i = ww$) \\
\hline
$q$ & Probability that two parasites cotransmit from intermediate hosts to definitive hosts \\
\hline
$\beta_i$ & Transmission rate of parasites from intermediate hosts to definitive hosts \\
\hline
$f_i$ & Reproduction rate of parasites in singly infected definitive hosts ($i = w$) or doubly infected hosts ($i = ww$)\\
\hline
$\delta$ & Natural death rate of parasites in the environment \\
\hline
\end{tabular}
\caption{Description of variables and parameters}
\label{table:varpardescription}
\end{table}

Here, we focus on the manipulation that enhance transmission from intermediate hosts to definitive hosts, we simplify the transmission from the parasite pool to intermediate hosts, such that no sequential infection occurs at this transmission state. 
Sequential infection can happen when parasites transmit from intermediate hosts to definitive hosts. 
Therefore, a singly infected definitive host can be further infected by another parasite if it consumes infected intermediate hosts. 
The dynamics of the system are illustrated in figure (\ref{fig:schematic}).

\begin{figure}[ht!]
\includegraphics[width=\textwidth]{Figures/schematic.jpeg}
\caption{Schematic of the model. The compartments of intermediate hosts are in the blue box, including $I_s$ (susceptible host), $I_w$ (singly infected host), and $I_{ww}$ (doubly infected host). The compartments of definitive hosts are in the red box, including $D_s$ (susceptible host), $D_w$ (singly infected host), and $D_{ww}$ (doubly infected host). Definitive hosts prey upon intermediate host and infection happens when infected intermediate hosts are eaten by definitive hosts. 
$W$ represents the parasite pool in the environment where parasites are released from the definitive hosts.
\cha{The $W$ goes to to the Intermediate then? I would highlight the W as environment also in the figure.}
}
\label{fig:schematic}
\end{figure}

\subsection*{Basic reproduction ratio $R_0$}

\cha{What does it mean by a basic reproductive ratio?.}
The basic reproduction ratio $R_0$ (or basic reproduction number as often used in epidemiology) is an indication for parasite fitness. It can be understood as the expected number of offspring a parasite produces during its lifetime when introduced in a susceptible host population. We calculate the basic reproduction ratio $R_0$ using the next generation method (ref) (details are in supplementary).  

\begin{align}
R_0 = & \gamma I_s^* \frac{ p q \beta_{ww}}{\alpha_{ww} + d + P_{ww}} \frac{D_s^*}{\mu +\sigma_{ww}} \frac{f_{ww}}{\delta +\gamma I_s^*} + \nonumber \\
& \gamma  I_s^* \left( \frac{ (1-p)\beta_w}{\alpha_w + d + P_w} + \frac{2 p (1-q) \beta_{ww}}{\alpha_{ww} + d + P_{ww}} \right) \frac{D_s^*}{\mu + \sigma_w} \frac{f_w}{\delta +\gamma  I_s^*}
\end{align}

where $I_s^*$ and $D_s^*$ are the density of susceptible intermediate and definitive hosts at the disease-free equilibrium. 
Here, the expression of $R_0$ contains possible reproduction routes of a parasite, which can be via double or single infections. 
The first component corresponds to the double infections route, in which the focal parasite co-transmit with another parasite into a susceptible intermediate host, then co-transmits into a susceptible definitive host and reproduces. 
The second component corresponds to the single infection route, wherein the focal parasite infects a susceptible intermediate host via singly or doubly infections. 
It then transmits alone into the susceptible definitive host and eventually reproduces. 
In a disease-free environment, parasites are so rare that the reproduction ratio compartments with sequential infections is eliminated. 


If $R_0 > 1$, a parasite spreads when it is introduced into the disease-free equilibrium of prey and predator.
Intuitively, the higher the density of susceptible intermediate and definitive hosts, the bigger the value of $R_0$ as the reservoir for infection is larger. In contrast, regardless of the explicit form of the predation function, the higher the predation rate $P_w$ and $P_{ww}$, the lower the value of $R_0$ because the reservoir of intermediate hosts is smaller. 
The effect of host manipulation on the value of $R_0$ is not so straight forward because as host manipulation becomes more efficient, transmission rate from intermediate host to definitive host increases, but so does the predation rate. Higher predation rate results in smaller intermediate host reservoir for parasites to infect. To understand the effect of manipulation on the fitness of parasites and the ecological dynamics of the system, we need to specify the predation functions. 
\cha{Here we need to motivate the importance of knowing the form of the functions.}
For simplicity, we consider linear functions for predation 

\begin{align*}
& P_s(D_s + D_w + D_{ww}) = \rho D_{total}  \\
& P_w(D_s, D_w, D_{ww}, \beta_w) = (\rho + \beta_w) D_{total} \\
& P_{ww}(D_s, D_w, D_{ww}, \beta_{ww}) =  (\rho + \beta_{ww})D_{total}
\end{align*}

where $D_{total} = D_s + D_w + D_{ww}$ is the total density of the definitive hosts, and $\rho$ is the baseline capture rate of the predator on the prey. If an intermediate hosts is infected, it is captured by the definitive hosts with rate $\rho + \beta_w$ if it is singly infected, and with rate $\rho + \beta_{ww}$ if it is doubly infected. Zero values for $\beta_w$ and $\beta_{ww}$ suggest no manipulation. 

For simplicity, we also consider a linear function of the birth of definitive hosts

\begin{align*}
B(D_s, D_w, D_{ww}, I_s, I_w, I_{ww}) = \rho c D_{total} I_{total}
\end{align*}

where c is the efficiency of converting preys into predator's offspring, and $I_{total} = I_s + I_w + I_{ww}$ is the total density of the intermediate hosts.
The birth rate of the predators depends on the capture rate, but it is not affected by host manipulation as there is no supporting evidence to our best knowledge.

The explicit form of $I_s^*$ and $D_s^*$ depends on the explicit form of all birth and predation functions $B, R, P_s, P_w$ and $P_{ww}$, but it does not depends on the manipulation ability, or any other parameter of the parasite. $I_s^*$ and $D_s^*$ simply represents the prey-predator dynamics when there is no parasite. Given that the birth rate of the predator and the predation rate are linear functions with respect to the prey and predator density, the form of the birth rate $R$ of the prey therefore has a large effect on the susceptible intermediate and definitive host dynamics. 

\cha{The motivation for the linear and non-linear functions and a paragraph discussing that before we move to the next sections.}

\subsection*{Linear birth function of intermediate hosts}
Here, we consider the system when the birth function $R$ of the intermediate host is linear, specifically, $R(I_s, I_w, I_{ww}) = r I_{total}$. 
The equilibrium of intermediate and definitive hosts in the disease-free state are,
%
\begin{align*}
& I_{s0}^* = \frac{\mu}{c \rho} \\
& D_{s0}^* = \frac{r - d}{\rho}
\end{align*}
%
This equilibrium is always unstable. 
In particular, it has a cyclic behaviour because, at this equilibrium, the jacobian matrix of the system (\ref{odes:ihosts}, \ref{odes:dhosts}, \ref{odes:eparasite}) always has one imaginary eigenvalue with a positive real part. 
This follows from the Lotka-Voltera system using linear functions for prey birth and predation (reference...). \cha{Use for motivation above.}
Because the disease-free dynamics is cyclic, it is difficult to analyse the spread of a parasite (often evaluated when the disease-free state is stable). 
Here,  $R_0 > 1$  happens when the transmission rate from the environment to intermediate hosts $\gamma$ and the reproduction of the parasites $f_w$ and $f_{ww}$ has to be greater than a threshold (see supplementary ). 
However, even when this condition is satisfied, the parasite may not be able to spread and persist in a cyclic susceptible host dynamics (Figure \ref{fig:diseasefree:linear}). This result indeed agrees with the conclusion in \cite{Ripa:Evol:2013}, which suggests that it is difficult for a mutant to invade a cyclic population. 
In our case, it is not the invasion of a mutant but the spread of a parasite in a cyclic disease-free host population, but the argument remains valid in both cases. 
This issue deserves a more thorough investigation, which is out of the scope of this article. 
We therefore chose a non-linear birth function of the intermediate hosts to obtain a stable disease circulation state and focus on the effect of host manipulation on the ecological dynamics. 

\begin{figure}
\includegraphics[width=\textwidth]{Figures/diseasefree_linear.pdf}
\caption{Disease-free equilibrium using linear birth function. Solid gray line indicate the density of free-living parasites, blue lines indicate infected intermediate hosts while red lines indicate infected definitive hosts. Dashed lines indicate singly infected hosts while dot-dashed lines indicate doubly infected hosts. Parameter values  $\rho = 1.2, \  d = 0.9, \  r = 2.5, \ \gamma = 2.9, \ \alpha_w =  \alpha_{ww} =  0, \ \beta_w  = 1.5, \ \beta_{ww} = 1.5, \ p = 0.1,  \ c = 1.4, \ \mu = 0.9,  \ \sigma_w = \sigma_{ww} = 0, \ q = 0.01, \  f_w = 6.5, \  f_{ww} = 7.5, \ \delta = 0.9$, $R_0 = 2.233$ } 
\label{fig:diseasefree:linear}
\end{figure}

\subsection*{Non-linear birth function of intermediate hosts}
The non-linear birth function of intermediate hosts is as followed,
\begin{align*}
R(I_w, I_s,I_{ww}) = r I_{total} (1 - k I_{total})
\end{align*}
%
where $k$ is the intraspecific competition coefficient. 
The disease-free equilibrium is as follows
%
\begin{align*}
& I_{s0}^* = \frac{\mu}{c \rho } \\
& D_{s0}^* = \frac{c \rho  (r-d) - k \mu  r}{c \rho ^2}
\end{align*}
%
This equilibrium is stable if,
%
\begin{align*}
& r > d \\
& \frac{2 c \rho  \left(\sqrt{\frac{-d+\mu +r}{\mu }}-1\right)}{r}\leq k < \frac{c \rho  (r-d)}{\mu  r} \\
& \mu >\frac{4 c^2 \rho ^2 r - 4 c^2 d \rho ^2}{4 c k \rho r + k^2 r^2}
\end{align*}

The above conditions suggest that the intrinsic reproduction of intermediate hosts $r$ needs to be greater than their natural mortality rate $d$. 
More importantly, the intraspecific competition coefficient has to be within a range allowing the population to survive.
Finally, the definitive host's natural mortality rate must be sufficiently large. 
Satisfying such conditions, we obtain a stable disease-free equilibrium (Figure \ref{fig:ecotraject:nonlinear}B).

\begin{figure}[!ht]
\includegraphics[width=\textwidth]{Figures/ecotraject_nonlinear.jpeg}
\caption{A) Trajectories of parasite and host dynamics. The host includes both intermediate (blue) and definitive (orange) ones. A) Disease free equilibrium where parasite densities is zero. B) Disease stable equilibrium where there are multiple parasite densities which correspond to free parasite pool, singly infected hosts and doubly infected hosts. Parameters for disease free equilibrium $\rho =  1.2, \ d = 0.9, \  r = 2.5, \ \gamma =  2.9, \alpha_w = \alpha_{ww} =  0, \ \beta_w = \beta_{ww} = 1.5, \ p = 0.1, \  c = 1.4, \ \mu = 3.9, \ \sigma_w = \sigma_{ww} = 0, \ q = 0.01, \ f_w = f_{ww} = 7.5, \ \delta = 0.9, \ k = 0.26$. Disease stable equilibrium have the same parameter values except for higher host manipulation $ \beta_w =  \beta_{ww} = 4.5$ and parasite reproduction $ f_w  = f_{ww} = 45$}
\label{fig:ecotraject:nonlinear}
\end{figure}

When a parasite is introduced in the disease-free equilibrium, it spreads if its reproduction ratio $R_0 > 1$. 
Since the expression is complicated, we could not obtain solutions for this inequality without assumptions. 
Assuming that double infections and single infection result in the same parasite virulence ($\alpha_w = \alpha_{ww}$, $\sigma_w = \sigma_{ww}$), and reproduction in single infection is a linear function with respect to reproduction in double infections ($f_w = \epsilon f_{ww}$), we found that the parasite can establish and spread in the hosts if its reproduction value in single infection $f_w$ is greater than a threshold (Figure \ref{fig:bistability}, supplementary ). When $\epsilon > 1$, reproduction in double infections is greater than reproduction in single infection, whereas $\epsilon \leq 1$, reproduction in double infections is lower or equal to reproduction in single infection.
 
Our numerical results show that the parasite reproduction is substantial compared to other parameters (its value is 40 times greater than other parameters), suggesting that trophically transmitted parasites must release many offspring into the environment to persist. 
Interestingly, bistability occurs if the reproduction rate of the parasite in double infections is greater than in the single infection state. 
The parasite population will crash if it is disturbed and become too small (Figure \ref{fig:bistability}A, B). 

\begin{figure}[!ht]
\includegraphics[width = \textwidth]{Figures/bistability.jpeg}
\caption{Effect of parasite reproduction on the ecological dynamics. A, B) When reproduction of parasites in singly infected hosts is four times greater than those in doubly infected hosts $\epsilon = 4$. C, D) When reproduction of parasites are the same in singly and doubly infected hosts $\epsilon = 1$. Filled circles indicate stable equilibrium and open circles indicate unstable equilibrium. Parameter $\rho = 1.2, \  d = 0.9, \  r = 2.5, \ \gamma = 2.9, \ \alpha_w = 0, \ \alpha_ww =  0, \ \beta_w = 1.5, \ \beta_{ww} = 1.5, \ p = 0.1, \  c = 1.4, \ \mu = 3.9,  \ \sigma_w = 0, \ \sigma_{ww} = 0, \  q = 0.01, \ \delta = 0.9, \ k = 0.26$
}
\label{fig:bistability}
\end{figure}

\subsection*{The effect of host manipulation on ecological dynamics}

Host manipulation can be cooperative, that is, two parasites increase the predation rate on intermediate hosts, or $\beta_{ww} > \beta_w$. It can also be uncooperative, that is, predation rate on doubly-infected  intermediate hosts is lower than that on singly-infected ones, or $\beta_{ww} < \beta_w$.
Cooperation in parasite manipulation increases the basic reproduction ratio of the parasite. However, if the ability to manipulate host in single infection is not strong enough, such cooperation widen the bistable state of the system (Figure \ref{fig:manipR0}). Within the area of bistability, the basic reproduction ratio is less than one, suggesting that co-infected parasites that are cooperative at an intermediate level, yet have weak manipulative ability when alone, cannot establish a population. Parasites who can persist in the population are those that may have weak manipulative activity in single infection but become much more manipulative in coinfection. Likewise, parasites can persist if they are uncooperative but can manipulate the intermediate hosts effectively when alone.

\begin{figure}
\centering
\includegraphics[width=0.75\textwidth]{Figures/manip_bifur_R0.jpeg}
\caption{A). Bistability region when cooperation in manipulation is high (dark green area). As manipulation in single infection increases, the system only has one stable equilibrium (light green area). On the black line, manipulation is indifference between single infection and double infection. B) Basic reproduction ratio $R_0$ with respect to manipulation in single and double infection. $R_0 < 1$ indicates that the parasite cannot establish in a disease free prey-predator population.}
\label{fig:manipR0}
\end{figure}

Cooperation between parasites need not be limited to host manipulation. In fact, parasites can cooperate to have a higher reproduction rate when they are co-infected. In our model, it specifically indicates that parasites in doubly infected definitive hosts have higher reproduction rate than parasites in singly infected definitive hosts. Without any assumption on the relationship between manipulative ability and reproduction, parasites can be cooperative in both manipulation and reproduction. 
Interestingly, higher cooperation in both manipulation and reproduction enlarges the area of bistability despite the fact that it also shrinks the area of extinction (Figure \ref{fig:manipbifur}). This suggests that systems, in which parasites have much higher manipulative ability and reproduction rate when coinfected than when singly infected, are more prone to instability than systems with parasites that are less cooperative, or systems with parasites that sabotage each other in coinfection. In other words, having the best of both words at the individual level may not benefit the population as a whole.

\begin{figure}[!ht]
\centering
\includegraphics[scale=0.3]{Figures/manip_bifurcation.jpeg}
\caption{Bifurcation graph of $\beta_w$ (manipulation in singly infected host) and $\beta_{ww}$ (manipulation in doubly infected host) when cotransmission from the parasite pool to intermediate host is small $p = 0.01$ (A) and big $p = 0.5$. Bistability occurs in the shaded areas. Manipulation is indifference between single infection and double infection on the black thick line. Common parameter:  $\rho = 1.2, \ d = 0.9, \ r = 2.5, \ \gamma = 2.9, \ \alpha_w = 0, \ \alpha_{ww} = 0, \ p = 0.1, \ c = 1.4, \ \mu = 3.9, \ \sigma_w = 0, \ \sigma_ww = 0, \ q = 0.01, \ \delta = 0.9, \ k = 0.26, \ \epsilon = 0.5$. parameters for the thick boundary A) $\epsilon = 0.5, f_w = 36$, B) the dashed boundary $\epsilon = 1, f_w = 36$), and C) the dot-dashed boundary $\epsilon = 2, f_w = 35$.}
\label{fig:manipbifur}
\end{figure}

Increasing cotransmission probability $p$ from the parasite pool to intermediate hosts reduce the extinction area whereas increasing the cotransmission probability $q$ from intermediate hosts to definitive hosts broaden this area. When $p$ is high, doubly infected intermediate hosts are more abundant so that cooperation in host manipulation need not be too high to bring the population out of the bistability state. However, it also means that the singly infected intermediate hosts are few and parasites in single infection need to make more manipulative effort to successfully transmit (Figure \ref{fig:manipbifur}B). 
When $q$ is high, successful transmission to definitive hosts relies principally on the predation of susceptible definitive hosts on doubly infected intermediate hosts. Cooperation in manipulation therefore need to be sufficiently high to avoid bistability. Sequential transmission is also rarer because the probability of single infection $1-q$ is low. If the number of doubly infected intermediate hosts is low, transmission from intermediate hosts to definitive hosts is limited in general, which explains the widen extinction area.

\section*{Discussion}


\section*{Conclusion}


%%%%%%%%%%%%%%%%%%%%%
% Acknowledgments
%%%%%%%%%%%%%%%%%%%%%
% You may wish to remove the Acknowledgments section while your paper 
% is under review (unless you wish to waive your anonymity under
% double-blind review) if the Acknowledgments reveal your identity.
% If you remove this section, you will need to add it back in to your
% final files after acceptance.

 \section*{Acknowledgments}


 \section*{Statement of Authorship}
 
\section*{Data and Code Availability}
All data and simulation codes for generating figures are available on 

%All data and simulation codes for generating figures are available on \href{https://anonymous.4open.science/r/genedrives_mating-6841/}{Github}   
%(\url{https://anonymous.4open.science/r/genedrives_mating-6841/}).


\newpage{}

%% Reseting the counters for equations, figures, tables
\renewcommand{\theequation}{A\arabic{equation}}
% redefine the command that creates the equation number.
\renewcommand{\thetable}{A\arabic{table}}
\renewcommand{\thefigure}{A\arabic{figure}}

\setcounter{figure}{0}
\setcounter{equation}{0}  % reset counter 
\setcounter{table}{0}

\section*{Appendix A}


\section*{Appendix B}


\newpage{}

%%%%%%%%%%%%%%%%%%%%%
% Bibliography
%%%%%%%%%%%%%%%%%%%%%
% You can either type your references following the examples below, or
% compile your BiBTeX database and paste the contents of your .bbl file
% here. The amnatnat.bst style file should work for this---but please
% let us know if you run into any hitches with it!
%
% If you upload a .bib file with your submission, please upload the .bbl
% file as well; this will be required for typesetting.
%
% The list below includes sample journal articles, book chapters, and
% Dryad references.

\bibliographystyle{amnatnat}
%\bibliography{\string~/Bibtex/et.bib}

\begin{thebibliography}{24}
\providecommand{\natexlab}[1]{#1}

\bibitem[{Alizon (2012)}]{Alizon2012}
Samuel Alizon. 2012.
\newblock Parasite co-transmission and the evolutionary epidemiology of
  virulence.
\newblock {\em Evolution}, 67(4):921--933, November 2012.

\bibitem[{Alizon et~al.(2013)Alizon, de~Roode, Michalakis}]{Alizon2013}
Samuel Alizon, Jacobus~C. de~Roode, and Yannis Michalakis. 2013.
\newblock Multiple infections and the evolution of virulence.
\newblock {\em Ecology Letters}, 16(4):556--567, January 2013.

\bibitem[{Alizon and van Baalen(2008)}]{Alizon2008}
Samuel Alizon and Minus van Baalen. 2008.
\newblock Multiple infections, immune dynamics, and the evolution of virulence.
\newblock {\em The American Naturalist}, 172(4):E150--E168, October 2008.

\bibitem[{Benesh(2016)}]{Benesh:2016dj}
Daniel~P Benesh. 2016.
\newblock {Autonomy and integration in complex parasite life cycles.}
\newblock {\em Parasitology}, 143(14):1824 -- 1846, 2016.

\bibitem[{Choisy and de~Roode(2010)}]{Choisy2010}
Marc Choisy and Jacobus~C. de~Roode. 2010.
\newblock Mixed infections and the evolution of virulence: Effects of resource
  competition, parasite plasticity, and impaired host immunity.
\newblock {\em The American Naturalist}, 175(5):E105--E118, May 2010.

\bibitem[{Fenton and Rands(2006)}]{Fenton2006}
A.~Fenton and S.~A. Rands. 2006.
\newblock The impact of parasite manipulation and predator foraging behavior on
  predator - prey communitites.
\newblock {\em Ecology}, 87(11):2832--2841, November 2006.

\bibitem[{Gandon(2018)}]{Gandon2018}
Sylvain Gandon. 2018.
\newblock Evolution and manipulation of vector host choice.
\newblock {\em The American Naturalist}, 192(1):23--34, July 2018.

\bibitem[{Hadeler and Freedman(1989)}]{Hadeler1989}
K.~P. Hadeler and H.~I. Freedman. 1989.
\newblock Predator-prey populations with parasitic infection.
\newblock {\em Journal of Mathematical Biology}, 27(6):609--631, November 1989.

\bibitem[{Hafer and Milinski(2015)}]{Hafer:2015gl}
Nina Hafer and Manfred Milinski. 2015.
\newblock {When parasites disagree: evidence for parasite-induced sabotage of
  host manipulation.}
\newblock {\em Evolution}, 69(3):611 -- 620, 2015.

\bibitem[{Hosack et~al.(2008)Hosack, Rossignol, and van~den Driessche}]{Hosack2008}
Geoffrey~R. Hosack, Philippe~A. Rossignol, and P.~van~den Driessche. 2008.
\newblock The control of vector-borne disease epidemics.
\newblock {\em Journal of Theoretical Biology}, 255(1):16--25, November 2008.

\bibitem[{Hughes(2012)}]{Hughes2012}
David~P Hughes, Jacques Brodeur, and Frederic Thomas. 2012.
\newblock {\em Host Manipulation by Parasites}.
\newblock Oxford University Press, London, England, June 2012.

\bibitem[{Iritani and Sato(2018)}]{Iritani2018}
Ryosuke Iritani and Takuya Sato. 2018.
\newblock Host-manipulation by trophically transmitted parasites: The
  switcher-paradigm.
\newblock {\em Trends in Parasitology}, 34(11):934--944, November 2018.

\bibitem[{Molyneux and Jefferies(1986)}]{molyneux1986}
D.~H. Molyneux and D.~Jefferies. 1986.
\newblock Feeding behaviour of pathogen-infected vectors.
\newblock {\em Parasitology}, 92(3):721–736, 1986.

\bibitem[{Parker et~al.(2003)Parker, Chubb, Roberts, Michaud, and Milinski}]{Parker2003}
G.~A. Parker, J.~C. Chubb, G.~N. Roberts, M.~Michaud, and M.~Milinski. 2003.
\newblock Optimal growth strategies of larval helminths in their intermediate
  hosts.
\newblock {\em Journal of Evolutionary Biology}, 16(1):47--54, January 2003.

\bibitem[{Ripa and Dieckmann(2013)}]{Ripa:Evol:2013}
Jörgen Ripa and Ulf Dieckmann. 2013.
\newblock Mutant invasions and adaptive dynamics in variable environments.
\newblock {\em Evolution}, 67(5):1279--1290, 2013.

\bibitem[{Rogawa et~al.(2018)Rogawa, Ogata, and Mougi}]{Rogawa2018}
Akiyoshi Rogawa, Shigeki Ogata, and Akihiko Mougi. 2018.
\newblock Parasite transmission between trophic levels stabilizes
  predator{\textendash}prey interaction.
\newblock {\em Scientific Reports}, 8(1), August 2018.

\bibitem[{Roger and Bates(2007)}]{Rogers2007}
Matthew~E Rogers and Paul~A Bates. 2007.
\newblock Leishmania manipulation of sand fly feeding behavior results in
  enhanced transmission.
\newblock {\em {PLoS} Pathogens}, 3(6):e91, 2007.

\bibitem[{Roosien et~al.(2013)Roosien, Gomulkiewicz, Ingwell, Bosque-Perez, Rajabaskar, and Eigenbrode}]{Roosien2013}
Bryan~K. Roosien, Richard Gomulkiewicz, Laura~L. Ingwell, Nilsa~A. 2013.
  Bosque-P{\'{e}}rez, Dheivasigamani Rajabaskar, and Sanford~D. Eigenbrode.
\newblock Conditional vector preference aids the spread of plant pathogens:
  Results from a model.
\newblock {\em Environmental Entomology}, 42(6):1299--1308, December 2013.

\bibitem[{Seppala and Jokela(2008)}]{Seppl2008}
Otto Seppala and Jukka Jokela. 2008.
\newblock Host manipulation as a parasite transmission strategy when
  manipulation is exploited by non-host predators.
\newblock {\em Biology Letters}, 4(6):663--666, August 2008.

\bibitem[{van Baalen and Sabelis(1995)}]{vanBaalen1995}
Minus van Baalen and Maurice~W. Sabelis. 1995.
\newblock The dynamics of multiple infection and the evolution of virulence.
\newblock {\em The American Naturalist}, 146(6):881--910, December 1995.

\bibitem[Vickery and Poulin(2009)]{Vickery2009}
William~L. Vickery and Robert Poulin. 2009.
\newblock The evolution of host manipulation by parasites: a game theory
  analysis.
\newblock {\em Evolutionary Ecology}, 24(4):773--788, November 2009.

\bibitem[Wedekind and Milinski(1996)]{Wedekind1996}
C.~Wedekind and M.~Milinski. 1996.
\newblock Do three-spined sticklebacks avoid consuming copepods, the first
  intermediate host of \textit{Schistocephalus solidus}? - an experimental
  analysis of behavioural resistance.
\newblock {\em Parasitology}, 112(4):371--383, April 1996.

\bibitem[{Zimmer(2001)}]{zimmer:book:2001}
Carl Zimmer. 2001.
\newblock {\em {Parasite Rex: Inside the Bizarre World of Nature's Most
  Dangerous Creatures}}.
\newblock Atria Books, 2001.

\end{thebibliography}

\newpage{}

\section*{Tables}
\renewcommand{\thetable}{\arabic{table}}
\setcounter{table}{0}

\begin{table}[h]
%\caption{Examples of natural drive elements with properties simulated in this study.}
%\label{Table:Driveelements}
%\centering
%\begin{tabular}{lllll}\hline
%Species & Drive element & Drive type \& parameters & Ecological properties
% \\ \hline
%\textit{Tribolium castaneum}  & Medea & Viability drive $d$ & Probably low k, highly \\
%  & & & polyandrous\\
 %   & & & \cite{pai:EntoEA:2020} \\
%\textit{Drosophila melanogaster}  & Selfish Segregation  & Distortion drive
%$p$,$c$ & Probably low k, Polyandry \\ 
%& Distorter (SD) & & with some re-mating \\
%& & & latency \cite{singh:CurrSci:2004}\\
%\textit{Mus musculus}  & t-haplotype & Distortion drive
%$p$,$c$ & Probably low k, \\ 
%& & & Polyandry ($r=2,$, \\ 
%& & & \cite{birand:MolEcol:2022}) \\ \hline 
%\end{tabular}
%\bigskip{}
%\\
%{\footnotesize Note: Table titles should be short. Further details should go in a `notes' area after the tabular environment, like this. $^a$ Published the first description of \textit{Dimetrodon}.}
\end{table}
%
\newpage{}

\section*{Figure legends}

\renewcommand{\thefigure}{\arabic{figure}}
\setcounter{figure}{0}
%%

%\begin{figure}[h!]
 %   \centering
% \includegraphics[width=0.65\columnwidth]{figure1.eps}
 %   \caption{\textbf{Pictorial representation of the three mating complexities: mate-choice, mating network, and mating system that can affect gene drive's population dynamics.}}
%    \label{fig:fig1}    
%\end{figure}
%%


% \subsection*{figure legends for Appendices}

\renewcommand{\thefigure}{A\arabic{figure}}
\setcounter{figure}{0}


\end{document}
